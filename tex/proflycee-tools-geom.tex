% proflycee-tools-geom.tex
% Copyright 2023  Cédric Pierquet
% This work may be distributed and/or modified under the
% conditions of the LaTeX Project Public License, either version 1.3
% of this license or (at your option) any later version.
% The latest version of this license is in
%   http://www.latex-project.org/lppl.txt
% and version 1.3 or later is part of all distributions of LaTeX
% version 2005/12/01 or later.

%%------PaveDroitTikZ
\defKV[paramspave]{%
	Largeur=\xdef\PFPaveLg{#1},%
	Profondeur=\xdef\PFPavePf{#1},%
	Hauteur=\xdef\PFPaveHt{#1},%
	Angle=\xdef\PFPaveAngl{#1},%
	Fuite=\xdef\PFPaveFuite{#1},%
	Sommets=\def\PFPaveSommets{#1},%
	Epaisseur=\xdef\PFPaveThick{#1}
}
\setKVdefault[paramspave]{%
	Aff=false,%
	Plein=false,%
	Largeur=2,%
	Profondeur=1,%
	Hauteur=1.25,%
	Angle=30,%
	Fuite=0.5,%
	Epaisseur=thick,%
	Sommets=A§B§C§D§E§F§G§H,%
	Cube=false,%
	Math=false
}

\newcommand\PaveTikz[1][]{%
	\useKVdefault[paramspave]%
	\setKV[paramspave]{#1}%
	\ifboolKV[paramspave]{Cube}
	{\xdef\PFPavePf{\PFPaveLg}%
		\xdef\PFPaveHt{\PFPaveLg}}
	{}
	\setsepchar{§}%
	\readlist*\PFListeSommets\PFPaveSommets
	\itemtomacro\PFListeSommets[1]\PaveA
	\itemtomacro\PFListeSommets[2]\PaveB
	\itemtomacro\PFListeSommets[3]\PaveC
	\itemtomacro\PFListeSommets[4]\PaveD
	\itemtomacro\PFListeSommets[5]\PaveE
	\itemtomacro\PFListeSommets[6]\PaveF
	\itemtomacro\PFListeSommets[7]\PaveG
	\itemtomacro\PFListeSommets[8]\PaveH
	%les nœuds du pave
	\coordinate (\PaveA) at (0,0) ;
	\coordinate (\PaveB) at ({\PFPaveLg},0) ;
	\coordinate (\PaveC) at ($(\PaveB) + ({\PFPaveAngl}:{\PFPaveFuite*\PFPavePf})$) ;
	\coordinate (\PaveD) at ($(\PaveA) + ({\PFPaveAngl}:{\PFPaveFuite*\PFPavePf})$) ;
	\coordinate (\PaveE) at ($(\PaveA) + (0,{\PFPaveHt})$) ;
	\coordinate (\PaveF) at ($(\PaveB) + (0,{\PFPaveHt})$) ;
	\coordinate (\PaveG) at ($(\PaveC) + (0,{\PFPaveHt})$) ;
	\coordinate (\PaveH) at ($(\PaveD) + (0,{\PFPaveHt})$) ;
	\ifboolKV[paramspave]{Aff}
		{\draw (\PaveA) node[below left] {\ifboolKV[paramspave]{Math}{$\PaveA$}{\PaveA}} ;
			\draw (\PaveB) node[below right] {\ifboolKV[paramspave]{Math}{$\PaveB$}{\PaveB}} ;
			\draw (\PaveC) node[above right] {\ifboolKV[paramspave]{Math}{$\PaveC$}{\PaveC}} ;
			\ifboolKV[paramspave]{Plein}
				{}
				{\draw (\PaveD) node[above left] {\ifboolKV[paramspave]{Math}{$\PaveD$}{\PaveD}} ;}
				\draw (\PaveE) node[below left] {\ifboolKV[paramspave]{Math}{$\PaveE$}{\PaveE}} ;
				\draw (\PaveF) node[below right] {\ifboolKV[paramspave]{Math}{$\PaveF$}{\PaveF}} ;
				\draw (\PaveG) node[above right] {\ifboolKV[paramspave]{Math}{$\PaveG$}{\PaveG}} ;
				\draw (\PaveH) node[above left] {\ifboolKV[paramspave]{Math}{$\PaveH$}{\PaveH}} ;}
		{}%on affiche rien
	\draw[\PFPaveThick,line join=bevel] (\PaveA)--(\PaveB)--(\PaveF)--(\PaveE)--cycle
	                                    (\PaveB)--(\PaveC)--(\PaveG)--(\PaveF)--cycle
	                                    (\PaveG)--(\PaveH)--(\PaveE) ;
	\ifboolKV[paramspave]{Plein}
		{}
		{\draw[dashed,\PFPaveThick,line join=bevel] (\PaveA)--(\PaveD)--(\PaveC)
		                                            (\PaveD)--(\PaveH) ;}
}

%%------TétraèdreTikZ
\defKV[paramstetra]{%
	Largeur=\xdef\PFTetraLg{#1},%
	Profondeur=\xdef\PFTetraPf{#1},%
	Hauteur=\xdef\PFTetraHt{#1},%
	Alpha=\xdef\PFTetraAlpha{#1},%
	Beta=\xdef\PFTetraBeta{#1},%
	Sommets=\def\PFTetraSommets{#1},%
	Epaisseur=\xdef\PFTetraThick{#1}
}
\setKVdefault[paramstetra]{%
	Aff=false,%
	Plein=false,%
	Largeur=4,%
	Profondeur=1.25,%
	Hauteur=3,%
	Alpha=40,%
	Beta=60,%
	Epaisseur=thick,%
	Sommets=A§B§C§D,%
	Math=false
}

\newcommand\TetraedreTikz[1][]{%
	\useKVdefault[paramstetra]%
	\setKV[paramstetra]{#1}%
	\setsepchar{§}%
	\readlist*\PFListeSommets\PFTetraSommets
	\itemtomacro\PFListeSommets[1]\TetraA
	\itemtomacro\PFListeSommets[2]\TetraB
	\itemtomacro\PFListeSommets[3]\TetraC
	\itemtomacro\PFListeSommets[4]\TetraD
	%les nœuds du tétraèdre
	\coordinate (\TetraA) at (0,0) ;
	\coordinate (\TetraB) at ($(\TetraA) + ({-\PFTetraAlpha}:{\PFTetraPf})$) ;
	\coordinate (\TetraC) at ({\PFTetraLg},0) ;
	\coordinate (\TetraD) at ($(\TetraA) + ({\PFTetraBeta}:{\PFTetraHt})$) ;
	\ifboolKV[paramstetra]{Aff}
	{\draw (\TetraA) node[left] {\ifboolKV[paramstetra]{Math}{$\TetraA$}{\TetraA}} ;
		\draw (\TetraB) node[below] {\ifboolKV[paramstetra]{Math}{$\TetraB$}{\TetraB}} ;
		\draw (\TetraC) node[right] {\ifboolKV[paramstetra]{Math}{$\TetraC$}{\TetraC}} ;
		\draw (\TetraD) node[above] {\ifboolKV[paramstetra]{Math}{$\TetraD$}{\TetraD}} ;}
		{}%on affiche rien
	\draw[\PFTetraThick,line join=bevel] (\TetraA)--(\TetraD)--(\TetraC)--(\TetraB)--cycle
						                 (\TetraD)--(\TetraB) ;
	\ifboolKV[paramstetra]{Plein}
		{}
		{\draw[dashed,\PFTetraThick,line join=bevel] (\TetraA)--(\TetraC) ;}
}

%%------MainLevée TikZ
\tikzset{%
	mainlevee/.style args={#1et#2}{decorate,decoration={random steps,segment length=#1,amplitude=#2}},
	mainlevee/.default={5mm et 0.6pt}
}

%%Equations Cartésiennes + Affichages coordonnées
\RequirePackage{nicematrix}
%\RequirePackage{ifpdf}

\NewDocumentCommand\AffCoeffSgn{ s O{} m m D<>{} }{%
	\IfStrEq{#5}{}%si argument vide, on convertit en fraction
		{%
			\xintifboolexpr{\xinteval{#3} == 0}%
				{}%on n'affiche rien si le coeff est nul
				{%sinon on teste >0 puis else
					\xintifboolexpr{\xinteval{#3} > 0}%
						{%
							\IfBooleanTF{#1}{}{+}%
							\xintifboolexpr{\xinteval{#3} == 1}%
								{#4}%
								{\ConversionFraction[#2]{#3}#4}%
						}%
						{%
							\xintifboolexpr{\xinteval{#3} == -1}%
								{-#4\relax}%
								{\ConversionFraction[#2]{#3}#4}%
						}%
				}%
		}%
		%sinon on met en brut
		{%
			#3#4%
		}%
}

\NewDocumentCommand\AffCoeffSgnSimpl{ s O{} m D<>{} }{%
	\IfStrEq{#4}{}%si argument vide, on convertit en fraction
		{%
			\xintifboolexpr{\xinteval{#3} == 0}%
				{}%on n'affiche rien si le coeff est nul
				{%sinon on teste >0 puis else
					\xintifboolexpr{\xinteval{#3} > 0}%
						{%
							\IfBooleanTF{#1}{}{+}%
							\xintifboolexpr{\xinteval{#3} == 1}%
								{+1}%
								{+\ConversionFraction[#2]{#3}}%
						}%
						{%
							\xintifboolexpr{\xinteval{#3} == -1}%
								{-1}%
								{\ConversionFraction[#2]{#3}}%
					}	%
				}%
		}%
	%sinon on met en brut
		{%
			#3%
		}%
}

\defKV[eqcartplan]{%
	OptionCoeffs=\def\eqcartplformat{#1},%
	Facteur=\def\eqcartplfact{#1}
}

\setKVdefault[eqcartplan]{%
	OptionCoeffs={d},%
	SimplifCoeffs=false,%
	Facteur=1
}

\NewDocumentCommand\TrouveEqCartPlan{ }{%
	\begingroup
	\catcode`\;12
	\TrouveEqCartPlanAux
}

\NewDocumentCommand\TrouveEqCartPlanAux{ O{} r() r() d() }{%test commande générique avec VP ou PPP ou PVV
	\endgroup
	\restoreKV[eqcartplan]% revenir au valeurs par défaut
	\setKV[eqcartplan]{#1}% lit les arguments optionnels
	\IfNoValueTF{#4}%c'est Vect+Point
		{%
			\setsepchar{;}\readlist*\CoordVecNorm{#2}%
			\setsepchar{,}\readlist*\CoordPt{#3}%
			\itemtomacro\CoordVecNorm[1]\vecnx%
			\itemtomacro\CoordVecNorm[2]\vecny%
			\itemtomacro\CoordVecNorm[3]\vecnz%
			\itemtomacro\CoordPt[1]\xpta%
			\itemtomacro\CoordPt[2]\ypta%
			\itemtomacro\CoordPt[3]\zpta%
			%calculs
			\xdef\coeffd{-((\xpta)*(\vecnx)+(\ypta)*(\vecny)+(\zpta)*(\vecnz))}%
			\xdef\PPCMDenom{\xinteval{lcm([\xintDenominator{\xintIrr{\xinteval{\vecnx}}},\xintDenominator{\xintIrr{\xinteval{\vecny}}},\xintDenominator{\xintIrr{\xinteval{\vecnz}}},\xintDenominator{\xintIrr{\xinteval{\coeffd}}}])}}%
			\xdef\PGCDsiEntiers{1}%
			\xintifboolexpr{\xinteval{isint(\vecnx)}*\xinteval{isint(\vecny)}*\xinteval{isint(\vecnz)}*\xinteval{isint(\coeffd)} == 1}%tous les coeffs sont entiers
				{%
					\xdef\PGCDsiEntiers{\xinteval{gcd([\xinteval{\vecnx},\xinteval{\vecny},\xinteval{\vecnz},\xinteval{\coeffd}])}}%
				}%
				{}%
			%affichages
			\ifboolKV[eqcartplan]{SimplifCoeffs}%
				{%
					\AffCoeffSgn*[\eqcartplformat]{(\vecnx)*(\eqcartplfact)*\PPCMDenom/\PGCDsiEntiers}{x} \AffCoeffSgn[\eqcartplformat]{(\vecny)*(\eqcartplfact)*\PPCMDenom/\PGCDsiEntiers}{y} \AffCoeffSgn[\eqcartplformat]{(\vecnz)*(\eqcartplfact)*\PPCMDenom/\PGCDsiEntiers}{z} \AffCoeffSgnSimpl*[\eqcartplformat]{\coeffd*\PPCMDenom/\PGCDsiEntiers} = 0%
				}%
				{%
					\AffCoeffSgn*[\eqcartplformat]{\vecnx}{x} \AffCoeffSgn[\eqcartplformat]{\vecny}{y} \AffCoeffSgn[\eqcartplformat]{\vecnz}{z} \AffCoeffSgnSimpl*[\eqcartplformat]{\coeffd} = 0%
				}%
		}%sinon c'est Point+Point+Point ou vectdir+vectdir+point
		{%
			\IfSubStr{#2}{,}%c'est point+point+point
				{%
					\setsepchar{,}\readlist*\CoordPtA{#2}%
					\setsepchar{,}\readlist*\CoordPtB{#3}%
					\setsepchar{,}\readlist*\CoordPtC{#4}%
					\itemtomacro\CoordPtA[1]\ptxa%
					\itemtomacro\CoordPtA[2]\ptya%
					\itemtomacro\CoordPtA[3]\ptza%
					\itemtomacro\CoordPtB[1]\ptxb%
					\itemtomacro\CoordPtB[2]\ptyb%
					\itemtomacro\CoordPtB[3]\ptzb%
					\itemtomacro\CoordPtC[1]\ptxc%
					\itemtomacro\CoordPtC[2]\ptyc%
					\itemtomacro\CoordPtC[3]\ptzc%
					%calculs
					\xdef\vecxab{(\ptxb-\ptxa)}%
					\xdef\vecyab{(\ptyb-\ptya)}%
					\xdef\veczab{(\ptzb-\ptza)}%
					\xdef\vecxac{(\ptxc-\ptxa)}%
					\xdef\vecyac{(\ptyc-\ptya)}%
					\xdef\veczac{(\ptzc-\ptza)}%
					%coeffs a/b/c
					\xdef\coeffa{(\vecyab*\veczac-\veczab*\vecyac)}%
					\xdef\coeffb{(\vecxac*\veczab-\vecxab*\veczac)}%
					\xdef\coeffc{(\vecxab*\vecyac-\vecxac*\vecyab)}%
					%coeffd
					\xdef\coeffd{(-(\ptxa)*\vecyab*\veczac-(\ptza)*\vecxab*\vecyac-(\ptya)*\vecxac*\veczab+(\ptza)*\vecyab*\vecxac+(\ptxa)*\veczab*\vecyac+(\ptya)*\vecxab*\veczac)}%
					%pour simplifier
					\xdef\PPCMDenom{\xinteval{lcm([\xintDenominator{\xintIrr{\xinteval{\coeffa}}},\xintDenominator{\xintIrr{\xinteval{\coeffb}}},\xintDenominator{\xintIrr{\xinteval{\coeffc}}},\xintDenominator{\xintIrr{\xinteval{\coeffd}}}])}}%
					\xdef\PGCDsiEntiers{1}%
					\xintifboolexpr{\xinteval{isint(\coeffa)}*\xinteval{isint(\coeffb)}*\xinteval{isint(\coeffc)}*\xinteval{isint(\coeffd)} == 1}%tous les coeffs sont entiers
						{%
							\xdef\PGCDsiEntiers{\xinteval{gcd([\xinteval{\coeffa},\xinteval{\coeffb},\xinteval{\coeffc},\xinteval{\coeffd}])}}%
						}%
						{}%
					%affichages
					\ifboolKV[eqcartplan]{SimplifCoeffs}%
						{%
							\AffCoeffSgn*[\eqcartplformat]{\coeffa*(\eqcartplfact)*\PPCMDenom/\PGCDsiEntiers}{x} \AffCoeffSgn[\eqcartplformat]{\coeffb*(\eqcartplfact)*\PPCMDenom/\PGCDsiEntiers}{y} \AffCoeffSgn[\eqcartplformat]{\coeffc*(\eqcartplfact)*\PPCMDenom/\PGCDsiEntiers}{z} \AffCoeffSgnSimpl*[\eqcartplformat]{\coeffd*(\eqcartplfact)*\PPCMDenom/\PGCDsiEntiers} = 0%
						}%
						{%
							\AffCoeffSgn*[\eqcartplformat]{\coeffa}{x} \AffCoeffSgn[\eqcartplformat]{\coeffb}{y} \AffCoeffSgn[\eqcartplformat]{\coeffc}{z} \AffCoeffSgnSimpl*[\eqcartplformat]{\coeffd} = 0%
						}%
				}%
				{%
					\setsepchar{;}\readlist*\CoordVecA{#2}%
					\setsepchar{;}\readlist*\CoordVecB{#3}%
					\setsepchar{,}\readlist*\CoordPtC{#4}%
					\itemtomacro\CoordVecA[1]\vecxab%
					\itemtomacro\CoordVecA[2]\vecyab%
					\itemtomacro\CoordVecA[3]\veczab%
					\itemtomacro\CoordVecB[1]\vecxac%
					\itemtomacro\CoordVecB[2]\vecyac%
					\itemtomacro\CoordVecB[3]\veczac%
					\itemtomacro\CoordPtC[1]\ptxc%
					\itemtomacro\CoordPtC[2]\ptyc%
					\itemtomacro\CoordPtC[3]\ptzc%
					%coeff a/b/c
					\xdef\coeffa{((\vecyab)*(\veczac)-(\vecyac)*(\veczab))}%
					\xdef\coeffb{((\veczab)*(\vecxac)-(\veczac)*(\vecxab))}%
					\xdef\coeffc{((\vecxab)*(\vecyac)-(\vecxac)*(\vecyab))}%
					%coeffd
					\xdef\coeffd{-((\ptxc)*\coeffa+(\ptyc)*\coeffb+(\ptzc)*\coeffc)}%
					%pour simplifier
					\xdef\PPCMDenom{\xinteval{lcm([\xintDenominator{\xintIrr{\xinteval{\coeffa}}},\xintDenominator{\xintIrr{\xinteval{\coeffb}}},\xintDenominator{\xintIrr{\xinteval{\coeffc}}},\xintDenominator{\xintIrr{\xinteval{\coeffd}}}])}}%
					\xdef\PGCDsiEntiers{1}%
					\xintifboolexpr{\xinteval{isint(\coeffa)}*\xinteval{isint(\coeffb)}*\xinteval{isint(\coeffc)}*\xinteval{isint(\coeffd)} == 1}%tous les coeffs sont entiers
						{%
							\xdef\PGCDsiEntiers{\xinteval{gcd([\xinteval{\coeffa},\xinteval{\coeffb},\xinteval{\coeffc},\xinteval{\coeffd}])}}%
						}%
						{}%
					%affichages
					\ifboolKV[eqcartplan]{SimplifCoeffs}%
						{%
							\AffCoeffSgn*[\eqcartplformat]{\coeffa*(\eqcartplfact)*\PPCMDenom/\PGCDsiEntiers}{x} \AffCoeffSgn[\eqcartplformat]{\coeffb*(\eqcartplfact)*\PPCMDenom/\PGCDsiEntiers}{y} \AffCoeffSgn[\eqcartplformat]{\coeffc*(\eqcartplfact)*\PPCMDenom/\PGCDsiEntiers}{z} \AffCoeffSgnSimpl*[\eqcartplformat]{\coeffd*(\eqcartplfact)*\PPCMDenom/\PGCDsiEntiers} = 0%
						}%
						{%
							\AffCoeffSgn*[\eqcartplformat]{\coeffa}{x} \AffCoeffSgn[\eqcartplformat]{\coeffb}{y} \AffCoeffSgn[\eqcartplformat]{\coeffc}{z} \AffCoeffSgnSimpl*[\eqcartplformat]{\coeffd} = 0%
						}%
				}%
		}%
}

\defKV[eqcartdroite]{%
	OptionCoeffs=\def\eqcartdteformat{#1},%
	Facteur=\def\eqcartdtefact{#1}
}

\setKVdefault[eqcartdroite]{%
	OptionCoeffs={d},%
	SimplifCoeffs=false,%
	VectDirecteur=false,%
	Facteur=1
}

\NewDocumentCommand\TrouveEqCartDroite{ }{%
	\begingroup
	\catcode`\;12
	\TrouveEqCartDroiteAux
}

\NewDocumentCommand\TrouveEqCartDroiteAux{ O{} r() r() }{%vect/point ou point/point
	\endgroup
	\restoreKV[eqcartdroite]% revenir au valeurs par défaut
	\setKV[eqcartdroite]{#1}% lit les arguments optionnels
	%on teste si c'est point/point
	\IfSubStr{#2}{,}%c'est point+point, sinon c'est vecteur+point
		{%
			\setsepchar{,}\readlist*\CoordPtA{#2}%
			\setsepchar{,}\readlist*\CoordPtB{#3}%
			\itemtomacro\CoordPtA[1]\xpta%
			\itemtomacro\CoordPtA[2]\ypta%
			\itemtomacro\CoordPtB[1]\xptb%
			\itemtomacro\CoordPtB[2]\yptb%
			\xdef\vecnx{((\xptb)-(\xpta))}%
			\xdef\vecny{((\yptb)-(\ypta))}%
			%calculs
			\xdef\coeffd{((\xpta)*(\vecny)-(\ypta)*(\vecnx))}%
			\xdef\PPCMDenom{\xinteval{lcm([\xintDenominator{\xintIrr{\xinteval{\vecnx}}},\xintDenominator{\xintIrr{\xinteval{\vecny}}},\xintDenominator{\xintIrr{\xinteval{\coeffd}}}])}}%
			\xdef\PGCDsiEntiers{1}%
			\xintifboolexpr{\xinteval{isint(\vecnx)}*\xinteval{isint(\vecny)}*\xinteval{isint(\coeffd)} == 1}%tous les coeffs sont entiers
				{%
					\xdef\PGCDsiEntiers{\xinteval{gcd([\xinteval{\vecnx},\xinteval{\vecny},\xinteval{\coeffd}])}}%
				}%
				{}%
			%affichages
			\ifboolKV[eqcartdroite]{SimplifCoeffs}%
				{%
					\AffCoeffSgn*[\eqcartdteformat]{-\vecny*(\eqcartdtefact)*\PPCMDenom/\PGCDsiEntiers}{x} \AffCoeffSgn[\eqcartdteformat]{\vecnx*(\eqcartdtefact)*\PPCMDenom/\PGCDsiEntiers}{y} \AffCoeffSgnSimpl*[\eqcartdteformat]{\coeffd*(\eqcartdtefact)*\PPCMDenom/\PGCDsiEntiers} = 0%
				}%
				{%
					\AffCoeffSgn*[\eqcartdteformat]{-(\vecny)}{x} \AffCoeffSgn[\eqcartdteformat]{(\vecnx)}{y} \AffCoeffSgnSimpl*[\eqcartdteformat]{(\coeffd)} = 0%
				}%
		}
		{%
			\setsepchar{;}\readlist*\CoordVec{#2}%
			\setsepchar{,}\readlist*\CoordPt{#3}%
			\itemtomacro\CoordVec[1]\vecnx%
			\itemtomacro\CoordVec[2]\vecny%
			\itemtomacro\CoordPt[1]\xpta%
			\itemtomacro\CoordPt[2]\ypta%
			%calculs
			\ifboolKV[eqcartdroite]{VectDirecteur}%
				{%
					\xdef\coeffd{((\xpta)*(\vecny)-(\ypta)*(\vecnx))}%
				}%
				{%
					\xdef\coeffd{-((\xpta)*(\vecnx)+(\ypta)*(\vecny))}%
				}%
			\xdef\PPCMDenom{\xinteval{lcm([\xintDenominator{\xintIrr{\xinteval{\vecnx}}},\xintDenominator{\xintIrr{\xinteval{\vecny}}},\xintDenominator{\xintIrr{\xinteval{\coeffd}}}])}}%
			\xdef\PGCDsiEntiers{1}%
			\xintifboolexpr{\xinteval{isint(\vecnx)}*\xinteval{isint(\vecny)}*\xinteval{isint(\coeffd)} == 1}%tous les coeffs sont entiers
				{%
					\xdef\PGCDsiEntiers{\xinteval{gcd([\xinteval{\vecnx},\xinteval{\vecny},\xinteval{\coeffd}])}}%
				}%
				{}%
			%affichages
			\ifboolKV[eqcartdroite]{SimplifCoeffs}%
				{%
					\ifboolKV[eqcartdroite]{VectDirecteur}%
						{%
							\AffCoeffSgn*[\eqcartdteformat]{-(\vecny)*(\eqcartdtefact)*(\PPCMDenom)/(\PGCDsiEntiers)}{x} \AffCoeffSgn[\eqcartdteformat]{(\vecnx)*(\eqcartdtefact)*(\PPCMDenom)/(\PGCDsiEntiers)}{y} \AffCoeffSgnSimpl*[\eqcartdteformat]{(\coeffd)*(\eqcartdtefact)*(\PPCMDenom)/(\PGCDsiEntiers)} = 0%
						}%
						{%
							\AffCoeffSgn*[\eqcartdteformat]{(\vecnx)*(\eqcartdtefact)*(\PPCMDenom)/(\PGCDsiEntiers)}{x} \AffCoeffSgn[\eqcartdteformat]{(\vecny)*(\eqcartdtefact)*(\PPCMDenom)/(\PGCDsiEntiers)}{y} \AffCoeffSgnSimpl*[\eqcartdteformat]{(\coeffd)*(\eqcartdtefact)*(\PPCMDenom)/(\PGCDsiEntiers)} = 0%
						}%
				}%
				{%
					\ifboolKV[eqcartdroite]{VectDirecteur}%
						{%
							\AffCoeffSgn*[\eqcartdteformat]{-(\vecny)}{x} \AffCoeffSgn[\eqcartdteformat]{\vecnx}{y} \AffCoeffSgnSimpl*[\eqcartdteformat]{(\coeffd)} = 0%
						}%
						{%
							\AffCoeffSgn*[\eqcartdteformat]{\vecnx}{x} \AffCoeffSgn[\eqcartdteformat]{\vecny}{y} \AffCoeffSgnSimpl*[\eqcartdteformat]{(\coeffd)} = 0%
						}%
				}%
		}%%
}

\NewDocumentCommand\AffVecteur{ }{%
	\begingroup
	\catcode`\;12
	\AffVecteurAux
}

\NewDocumentCommand\AffVecteurAux{ O{d} D<>{} r() }{%
	\endgroup
	\setsepchar{;}\readlist*\CoordVec{#3}%
	\xintifboolexpr{\CoordVeclen == 2}%
		{%
			\IfSubStr{#1}{;}%
				{%
					\setsepchar{;}\readlist*\OptVec{#1}%
					\itemtomacro\OptVec[1]\optvecx%
					\itemtomacro\OptVec[2]\optvecy%
				}%
				{%
					\xdef\optvecx{#1}\xdef\optvecy{#1}%
				}%
			\itemtomacro\CoordVec[1]\vecx%
			\itemtomacro\CoordVec[2]\vecy%
			\begin{pNiceMatrix}[#2] \ConversionFraction[\optvecx]{\vecx} \\ \ConversionFraction[\optvecy]{\vecy} \end{pNiceMatrix}%
		}%
		{}%
	\xintifboolexpr{\CoordVeclen == 3}%
		{%
			\IfSubStr{#1}{;}%
				{%
					\setsepchar{;}\readlist*\OptVec{#1}%
					\itemtomacro\OptVec[1]\optvecx%
					\itemtomacro\OptVec[2]\optvecy%
					\itemtomacro\OptVec[3]\optvecz%
				}%
				{%
					\xdef\optvecx{#1}\xdef\optvecy{#1}\xdef\optvecz{#1}%
				}%
			\itemtomacro\CoordVec[1]\vecx%
			\itemtomacro\CoordVec[2]\vecy%
			\itemtomacro\CoordVec[3]\vecz%
			\begin{pNiceMatrix}[#2] \ConversionFraction[\optvecx]{\vecx} \\ \ConversionFraction[\optvecy]{\vecy} \\ \ConversionFraction[\optvecz]{\vecz} \end{pNiceMatrix}%
		}%
		{}%
}

\NewDocumentCommand\AffPoint{ O{d} r() }{%
	\setsepchar{,}
	\readlist*\CoordPt{#2}%
	\xintifboolexpr{\CoordPtlen == 2}%
		{%
			\IfSubStr{#1}{,}%si l'option est globale...
				{%
					\setsepchar{,}\readlist*\OptPt{#1}%
					\itemtomacro\OptPt[1]\optptx%
					\itemtomacro\OptPt[2]\optpty%
				}%
				{%
					\xdef\optptx{#1}\xdef\optpty{#1}%
				}%
			\itemtomacro\CoordPt[1]\ptx%
			\itemtomacro\CoordPt[2]\pty%
			\left( \ConversionFraction[\optptx]{\ptx} ; \ConversionFraction[\optpty]{\pty} \right)%
		}%
		{}%
	\xintifboolexpr{\CoordPtlen == 3}%
		{%
			\IfSubStr{#1}{,}%si l'option est globale...
				{%
					\setsepchar{,}\readlist*\OptPt{#1}%
					\itemtomacro\OptPt[1]\optptx%
					\itemtomacro\OptPt[2]\optpty%
					\itemtomacro\OptPt[3]\optptz%
				}%
				{%
					\xdef\optptx{#1}\xdef\optpty{#1}\xdef\optptz{#1}%
				}%
			\itemtomacro\CoordPt[1]\ptx%
			\itemtomacro\CoordPt[2]\pty%
			\itemtomacro\CoordPt[3]\ptz%
			\left( \ConversionFraction[\optptx]{\ptx} ; \ConversionFraction[\optpty]{\pty} ; \ConversionFraction[\optptz]{\ptz} \right)%
		}%
		{}%
}

%%Équation paramétrique de droite
\defKV[eqparamdroite]{%
	OptionCoeffs=\def\eqparamdteformat{#1},%
	Reel=\def\eqparamdtereel{#1}
}

\setKVdefault[eqparamdroite]{%
	OptionCoeffs={d},%
	Reel=k,%
	Aligne=false,%
	Oppose=false,%
	Rgras=false
}

\NewDocumentCommand\AffVarDteParam{ m m }{%
	\xdef\restmp{\ConversionFraction[\eqparamdteformat]{#1} \AffCoeffSgn[\eqcartdteformat]{#2}{\eqparamdtereel}}%
	\xintifboolexpr{\xinteval{#1} == 0 'and' \xinteval{#2} == 0}{\xdef\restmp{0}}{}%
	\xintifboolexpr{\xinteval{#1} == 0 'and' \xinteval{#2} != 0}{\xdef\restmp{\AffCoeffSgn*[\eqcartdteformat]{#2}{\eqparamdtereel}}}{}%
	\restmp%
}

\NewDocumentCommand\AffVarDteParamAlign{ m m }{%
	\xdef\restmp{\ConversionFraction[\eqparamdteformat]{#1} & \AffCoeffSgn[\eqcartdteformat]{#2}{\eqparamdtereel}}%
	\xintifboolexpr{\xinteval{#1} == 0 'and' \xinteval{#2} == 0}{\xdef\restmp{0 & }}{}%
	\xintifboolexpr{\xinteval{#1} == 0 'and' \xinteval{#2} != 0}{\xdef\restmp{ & \AffCoeffSgn*[\eqcartdteformat]{#2}{\eqparamdtereel}}}{}%
	\restmp%
}

\NewDocumentCommand\TrouveEqParamDroite{ }{%
	\begingroup
	\catcode`\;12
	\TrouveEqParamDroiteAux
}

\NewDocumentCommand\TrouveEqParamDroiteAux{ O{} r() r() }{%vect/point ou point/point
	\endgroup
	\restoreKV[eqparamdroite]% revenir au valeurs par défaut
	\setKV[eqparamdroite]{#1}% lit les arguments optionnels
	%on teste si c'est point/point
	\IfSubStr{#2}{,}%c'est point+point, sinon c'est vecteur+point
		{%
			\setsepchar{,}\readlist*\CoordPtA{#2}%
			\setsepchar{,}\readlist*\CoordPtB{#3}%
			\itemtomacro\CoordPtA[1]\xpta%
			\itemtomacro\CoordPtA[2]\ypta%
			\itemtomacro\CoordPtA[2]\zpta%
			\itemtomacro\CoordPtB[1]\xptb%
			\itemtomacro\CoordPtB[2]\yptb%
			\itemtomacro\CoordPtB[2]\zptb%
			\ifboolKV[eqparamdroite]{Oppose}%
				{%
					\xdef\vecdirx{((\xpta)-(\xptb))}%
					\xdef\vecdiry{((\ypta)-(\yptb))}%
					\xdef\vecdirz{((\zpta)-(\zptb))}%
				}%
				{%
					\xdef\vecdirx{((\xptb)-(\xpta))}%
					\xdef\vecdiry{((\yptb)-(\ypta))}%
					\xdef\vecdirz{((\zptb)-(\zpta))}%
				}%
		}%
		{%
			\setsepchar{;}\readlist*\CoordVec{#2}%
			\setsepchar{,}\readlist*\CoordPt{#3}%
			\itemtomacro\CoordVec[1]\vecdirx%
			\itemtomacro\CoordVec[2]\vecdiry%
			\itemtomacro\CoordVec[3]\vecdirz%
			\itemtomacro\CoordPt[1]\xpta%
			\itemtomacro\CoordPt[2]\ypta%
			\itemtomacro\CoordPt[3]\zpta%
		}%
	\ifboolKV[eqparamdroite]{Aligne}%
		{%
			\left\lbrace\begin{array}{@{\,}l@{\;=\;}l@{\;}r}
				x & \AffVarDteParamAlign{\xpta}{\vecdirx} \\
				y & \AffVarDteParamAlign{\ypta}{\vecdiry} \\
				z & \AffVarDteParamAlign{\zpta}{\vecdirz} \\
			\end{array}\right.
			\text{, } \eqparamdtereel \in \ifboolKV[eqparamdroite]{Rgras}{\textbf{R}}{\mathbb{R}}%
		}%
		{%
			\begin{dcases}
				x = \AffVarDteParam{\xpta}{\vecdirx} \\
				y = \AffVarDteParam{\ypta}{\vecdiry} \\
				z = \AffVarDteParam{\zpta}{\vecdirz} \\
			\end{dcases}
			\text{, } \eqparamdtereel \in \ifboolKV[eqparamdroite]{Rgras}{\textbf{R}}{\mathbb{R}}%
		}%
}

\NewDocumentCommand\TrouveDistancePtPlan{ }{%
	\begingroup
	\catcode`\;12
	\TrouveDistancePtPlanAux
}

\NewDocumentCommand\TrouveDistancePtPlanAux{ r() r() d() }{%pt+vect+pt
	\endgroup
	\IfNoValueTF{#3}%c'est Point + Equation // sinon c'est point + vectnorm + point
		{%
			\StrDel{#2}{=0}[\tmpeq]%
			%%\tmpeq \text{ et }%
			\setsepchar{,}\readlist*\CoordPtA{#1}%
			\itemtomacro\CoordPtA[1]\xa%
			\itemtomacro\CoordPtA[2]\ya%
			\itemtomacro\CoordPtA[3]\za%.
			%calcul de d
			\StrSubstitute{\tmpeq}{x}{(0)}[\tmpcoeffd]%
			\StrSubstitute{\tmpcoeffd}{y}{(0)}[\tmpcoeffd]%
			\StrSubstitute{\tmpcoeffd}{z}{(0)}[\tmpcoeffd]%
			\xdef\coeffd{\tmpcoeffd}%
			%%d=\xinteval{\coeffd} \text{ et }%
			%calcul de a
			\StrSubstitute{\tmpeq}{x}{(1)}[\tmpcoeffa]%
			\StrSubstitute{\tmpcoeffa}{y}{(0)}[\tmpcoeffa]%
			\StrSubstitute{\tmpcoeffa}{z}{(0)}[\tmpcoeffa]%
			\xdef\vectx{(\tmpcoeffa)-(\coeffd)}%
			%%a=\xinteval{\vectx} \text{ et }%
			%calcul de b
			\StrSubstitute{\tmpeq}{x}{(0)}[\tmpcoeffb]%
			\StrSubstitute{\tmpcoeffb}{y}{(1)}[\tmpcoeffb]%
			\StrSubstitute{\tmpcoeffb}{z}{(0)}[\tmpcoeffb]%
			\xdef\vecty{(\tmpcoeffb)-(\coeffd)}%
			%%b=\xinteval{\vecty} \text{ et }%
			%calcul de c
			\StrSubstitute{\tmpeq}{x}{(0)}[\tmpcoeffc]%
			\StrSubstitute{\tmpcoeffc}{y}{(0)}[\tmpcoeffc]%
			\StrSubstitute{\tmpcoeffc}{z}{(1)}[\tmpcoeffc]%
			\xdef\vectz{(\tmpcoeffc)-(\coeffd)}%
			%c=\xinteval{\vectz} \text{ et }%
			%calcul du numérateur
			\StrSubstitute{\tmpeq}{x}{(\xa)}[\resnum]%
			\StrSubstitute{\resnum}{y}{(\ya)}[\resnum]%
			\StrSubstitute{\resnum}{z}{(\za)}[\resnum]%
			%%\text{num}=\xinteval{\resnum} \text{ et }
			%le carré
			\xdef\restmp{(\resnum)**2/((\vectx)**2+(\vecty)**2+(\vectz)**2)}%
			%%\text{resultatcarré}=\xinteval{\restmp} \text{ et }%
		}%
		{%
			\setsepchar{,}\readlist*\CoordPtA{#1}%
			\setsepchar{;}\readlist*\CoordVec{#2}%
			\setsepchar{,}\readlist*\CoordPtB{#3}%
			\itemtomacro\CoordPtA[1]\xa%
			\itemtomacro\CoordPtA[2]\ya%
			\itemtomacro\CoordPtA[3]\za%
			\itemtomacro\CoordVec[1]\vectx%
			\itemtomacro\CoordVec[2]\vecty%
			\itemtomacro\CoordVec[3]\vectz%
			\itemtomacro\CoordPtB[1]\xb%
			\itemtomacro\CoordPtB[2]\yb%
			\itemtomacro\CoordPtB[3]\zb%
			\def\restmp{((\vectx)*((\xb)-(\xa))+(\vecty)*((\yb)-(\ya))+(\vectz)*((\zb)-(\za)))**2/((\vectx)**2+(\vecty)**2+(\vectz)**2)}%
		}%
	\SimplificationRacine{\restmp}%
}

\NewDocumentCommand\TrouveNorme{ }{%
	\begingroup
	\catcode`\;12
	\TrouveNormeAux
}

\NewDocumentCommand\TrouveNormeAux{ r() d() }{%pt+vect+pt
	\endgroup
	\IfNoValueTF{#2}%c'est Vecteur // sinon c'est point point
		{%
			\setsepchar{;}\readlist*\CoordVec{#1}%
			\xintifboolexpr{\CoordVeclen == 2}%
				{%
					\itemtomacro\CoordVec[1]\xveca%
					\itemtomacro\CoordVec[2]\yveca%
					\xdef\restmp{(\xveca)**2+(\yveca)**2}%
				}
				{}%
			\xintifboolexpr{\CoordVeclen == 3}%%
				{%
					\itemtomacro\CoordVec[1]\xveca%
					\itemtomacro\CoordVec[2]\yveca%
					\itemtomacro\CoordVec[3]\zveca%
					\xdef\restmp{(\xveca)**2+(\yveca)**2+(\zveca)**2}%
				}%
				{}%
		}%
		{%
			\setsepchar{,}\readlist*\CoordPtA{#1}%
			\setsepchar{,}\readlist*\CoordPtB{#2}%
			\xintifboolexpr{\CoordPtAlen == 2}%
				{%
					\itemtomacro\CoordPtA[1]\xa%
					\itemtomacro\CoordPtA[2]\ya%
					\itemtomacro\CoordPtB[1]\xb%
					\itemtomacro\CoordPtB[2]\yb%
					\xdef\restmp{((\xb)-(\xa))**2+((\yb)-(\ya))**2}%
				}%
				{}
			\xintifboolexpr{\CoordPtAlen == 3}%
				{%
					\itemtomacro\CoordPtA[1]\xa%
					\itemtomacro\CoordPtA[2]\ya%
					\itemtomacro\CoordPtA[3]\za%
					\itemtomacro\CoordPtB[1]\xb%
					\itemtomacro\CoordPtB[2]\yb%
					\itemtomacro\CoordPtB[3]\zb%
					\xdef\restmp{((\xb)-(\xa))**2+((\yb)-(\ya))**2+((\zb)-(\za))**2}
				}%
				{}%
		}%
	\SimplificationRacine{\restmp}%
}

\endinput