% proflycee-tools-recreat.tex
% Copyright 2023  Cédric Pierquet
% This work may be distributed and/or modified under the
% conditions of the LaTeX Project Public License, either version 1.3
% of this license or (at your option) any later version.
% The latest version of this license is in
%   http://www.latex-project.org/lppl.txt
% and version 1.3 or later is part of all distributions of LaTeX
% version 2005/12/01 or later.

%%------FENÊTRE CALCUL FORMEL
\newcommand\CFchap{\textasciicircum}
\newcounter{CFnum}
%def des clés
\defKV[paramfenxcas]{%
	Largeur=\def\CFlarg{#1},%
	EspaceLg=\def\CFesplg{#1},%
	PremCol=\def\CFpremcol{#1},%
	HautPremCol=\def\CFhpremcol{#1},%
	Taille=\def\CFtaille{#1},%
	Couleur=\def\CFcouleur{#1},%
	TailleTitre=\def\CFtailletitre{#1},%
	CouleurCmd=\def\CFcoulcmd{#1},%
	CouleurRes=\def\CFcoulres{#1},%
	PosCmd=\def\CFposcmd{#1},%
	PosRes=\def\CFposres{#1},%
	LabelTitre=\def\CFlabeltitre{#1}%
}
\setKVdefault[paramfenxcas]{%
	Largeur=16,EspaceLg=2pt,PremCol=0.3,HautPremCol=0.4,%
	Couleur=darkgray,Menu=true,Titre=false,TailleTitre=\normalsize,Taille=\normalsize,%
	Sep=true,PosRes=centre,PosCmd=gauche,%
	CouleurCmd=red,CouleurRes=blue,%
	LabelTitre={Résultats obtenus avec un logiciel de Calcul Formel}%
}

\newcommand\CalculFormelParametres[1][]{%
	\setcounter{CFnum}{0}
	\useKVdefault[paramfenxcas]%
	\setKV[paramfenxcas]{#1}% on paramètres les nouvelles clés et on les simplifie
}

\defKV[paramlgxcas]{%
	HautCmd=\def\CFhle{#1},%
	HautRes=\def\CFhlr{#1}
}
\setKVdefault[paramlgxcas]{%
	HautCmd=0.75,%
	HautRes=0.75
}

\newcommand\CalculFormelLigne[3][]{%
	\addtocounter{CFnum}{1}
	\def\CFL{\theCFnum}%
	\def\CFLA{\inteval{\CFL-1}}%
	\useKVdefault[paramlgxcas]%
	\setKV[paramlgxcas]{#1}% on paramètres les nouvelles clés et on les simplifie
	\def\CFLA{\inteval{\CFL-1}}%
	%DÉCLARATION DES NŒUDS (les "6" coins des lignes commande et résultat)
	\xintifboolexpr{\CFL == 1}%si c'est la première ligne
		{\coordinate (A0\CFL) at (0,0);}
		{\coordinate (A0\CFL) at ($(A2\CFLA) + (0,{-\CFesplg})$);}
	\coordinate (A1\CFL) at ($(A0\CFL) + (0,{-\CFhle})$);
	\coordinate (A2\CFL) at ($(A1\CFL) + (0,{-\CFhlr})$);
	\coordinate (A3\CFL) at ($(A0\CFL) + ({\CFlarg},0)$);
	\coordinate (A4\CFL) at ($(A1\CFL) + ({\CFlarg},0)$);
	\coordinate (A5\CFL) at ($(A2\CFL) + ({\CFlarg},0)$);
	%DÉCLARATION DES NŒUDS INTERMÉDIAIRES (pour les commandes et les résultats)
	\coordinate (C1\CFL) at ($(A0\CFL) + (0,{-0.5*\CFhle})$);
	\coordinate (C2\CFL) at ($(A0\CFL) + ({0.5*\CFlarg},{-0.5*\CFhle})$);
	\coordinate (C3\CFL) at ($(A0\CFL) + ({\CFlarg},{-0.5*\CFhle})$);
	\coordinate (R1\CFL) at ($(A1\CFL) + (0,{-0.5*\CFhlr})$);
	\coordinate (R2\CFL) at ($(A1\CFL) + ({0.5*\CFlarg},{-0.5*\CFhlr})$);
	\coordinate (R3\CFL) at ($(A1\CFL) + ({\CFlarg},{-0.5*\CFhlr})$);
	%RECTANGLE DE BASE
	\draw[\CFcouleur] (A0\CFL) rectangle (A5\CFL) ;
	%LA COMMANDE EN ROUGE
	\IfStrEq{\CFposcmd}{centre}%si poscmd=center
		{\draw (C2\CFL) node[\CFcoulcmd,font=\CFtaille] {#2} ;}
		{}
	\IfStrEq{\CFposcmd}{gauche}%si poscmd=left
		{\draw (C1\CFL) node[right,\CFcoulcmd,font=\CFtaille] {#2} ;}
		{}
	\IfStrEq{\CFposcmd}{right}%si poscmd=right
		{\draw (C3\CFL) node[left,\CFcoulcmd,font=\CFtaille] {#2} ;}
		{}
	%LE RÉSULTAT
	\IfStrEq{\CFposres}{centre}%si posrep=center
		{\draw (R2\CFL) node[\CFcoulres,font=\CFtaille] {#3} ;}
		{}
	\IfStrEq{\CFposres}{gauche}%si posrep=left
		{\draw (R1\CFL) node[right,\CFcoulres,font=\CFtaille] {#3} ;}
		{}
	\IfStrEq{\CFposres}{right}%si posrep=right
		{\draw (R3\CFL) node[left,\CFcoulres,font=\CFtaille] {#3} ;}
		{}
	\ifboolKV[paramfenxcas]{Sep}%si sep=true
		{\draw[\CFcouleur] (A1\CFL) -- (A4\CFL);}%
		{}
	%LE PETIT NUMÉRO
	\draw[\CFcouleur] (A0\CFL) rectangle ++ ({-\CFpremcol},{-\CFhpremcol}) node[\CFcouleur,midway,font=\small\sffamily\bfseries] {\CFL} ;
	%LE RECTANGLE "MENU"
	\ifboolKV[paramfenxcas]{Menu}%si menu=true
		{\draw[\CFcouleur,fill=\CFcouleur!25] (A5\CFL) rectangle ++ (-0.65,0.25) node[black,midway,font=\tiny\sffamily\bfseries] {MENU} ;}%
		{}
	%LE BLOC "TITRE"
	\ifboolKV[paramfenxcas]{Titre}%si titre=true
		{\draw[\CFcouleur,fill=lightgray!25,rounded corners] ($(A01) + (0,2pt)$) rectangle ++ ($({\CFlarg},2em)$) node[CouleurVertForet!50!\CFcouleur,midway,font=\CFtailletitre\ttfamily\bfseries] {\CFlabeltitre};}
		{}
}

%%------CartoucheCapytale
\definecolor{vertcapyt}{rgb}{0.0,0.5,0.0}
%\definecolor{vertcapyt}{HTML}{008000}
\DeclareTotalTCBox{\CartoucheCapytale}{ s O{} m }
	{enhanced,size=fbox,on line,arc=3pt,colback=vertcapyt,colframe=vertcapyt,fontupper=\IfBooleanTF{#1}{\ttfamily}{\sffamily}\bfseries,colupper=white}%
	{#3#2~{\scriptsize\faLink}}

%%------SUDOMATHS
\defKV[PLTIKZSUDOM]{%
	CouleurTexte=\def\PLSMcoultexte{#1},%
	Epaisseur=\def\PLSMepf{#1},%
	Epaisseurg=\def\PLSMepg{#1},%
	Unite=\def\PLSMunite{#1},%
	CouleurCase=\def\PLSMcoulcase{#1},%
	NbCol=\def\PLSMnbcol{#1},%
	NbSubCol=\def\PLSMnbsubcol{#1},%
	NbLig=\def\PLSMnblig{#1},%
	NbSubLig=\def\PLSMnbsublig{#1},%
	Police=\def\PLSMfonte{#1},%
	PoliceLeg=\def\PLSMfonteleg{#1},%
	ListeLegV=\def\PLSMlistelegv{#1},%
	ListeLegH=\def\PLSMlistelegh{#1},%
	DecalLegende=\def\PLSMdecalleg{#1}
}

\setKVdefault[PLTIKZSUDOM]{%
	Epaisseurg=1.5pt,%
	Epaisseur=0.5pt,%
	Unite=1cm,%
	CouleurCase=cyan!25,%
	CouleurTexte=blue,%
	NbCol=9,%
	NbSubCol=3,%
	NbLig=9,%
	NbSubLig=3,
	Police=\normalfont\normalsize,%
	PoliceLeg=\normalfont\sffamily,%
	Legendes=true,%
	ListeLegV=ABCDEFGHIJKLMNOPQRSTUVWXYZ,%
	ListeLegH=abcdefghijklmnopqrstuvwxyz,%
	DecalLegende=0.45
}

\NewDocumentEnvironment{EnvSudoMaths}{ O{} m }
	{
	\useKVdefault[PLTIKZSUDOM]
	\setKV[PLTIKZSUDOM]{#1}% on paramètres les nouvelles clés et on les simplifie
	%calculs intermédiaires
	\def\larcolinter{\inteval{\PLSMnbcol/\PLSMnbsubcol}}
	\def\larliginter{\inteval{\PLSMnblig/\PLSMnbsublig}}
	%lecture liste
	\IfEq{#2}{}{}%
		{%
			\setsepchar[.]{§./}%
			\readlist*\SPGrilleSudoMaths{#2}%
		}
	%débt envtik
	\begin{tikzpicture}[x=\PLSMunite,y=\PLSMunite,line join=miter]
		%cases
		\IfEq{#2}{}{}%
		{%
		\foreach \i in {1,2,...,\PLSMnblig}{%
			\foreach \j in {1,2,...,\PLSMnbcol}{%
				\itemtomacro\SPGrilleSudoMaths[\i,\j]\SMcase
				\IfSubStr{\SMcase}{*}%si on veut colorier via *
				{%
					\StrDel{\SMcase}{*}[\SMcaseb]%
					\draw[draw=none,fill=\PLSMcoulcase] ({\j-1},{1-\i}) rectangle++ (1,-1) node[inner sep=0pt,outer sep=0pt,\PLSMcoultexte,font=\PLSMfonte,midway] {\SMcaseb} ;%
				}
				{%
					\draw ({\j-0.5},{0.5-\i}) node[inner sep=0pt,outer sep=0pt,\PLSMcoultexte,font=\PLSMfonte] {\SMcase} ;%
				}
			}
		}%
		}
		%grilles
		\draw[line width=\PLSMepg] (0,0) rectangle ({\PLSMnbcol},{-\PLSMnblig}) ;
		\draw[line width=\PLSMepf,xstep=1,ystep=1] (0,0) grid ({\PLSMnbcol},{-\PLSMnblig}) ;
		\draw[line width=\PLSMepg,xstep=\larcolinter,ystep=\larliginter] (0,0) grid ({\PLSMnbcol},{-\PLSMnblig}) ;
		%légendes
		\ifboolKV[PLTIKZSUDOM]{Legendes}
			{%
				\foreach \i in {1,2,...,\PLSMnbcol}{\draw ({\i-0.5},{\PLSMdecalleg}) node[inner sep=0pt,outer sep=0pt,font=\PLSMfonteleg] {\strut\StrChar{\PLSMlistelegh}{\i}} ;}
				\foreach \j in {1,2,...,\PLSMnblig}{\draw ({-\PLSMdecalleg},{0.5-\j}) node[inner sep=0pt,outer sep=0pt,font=\PLSMfonteleg] {\StrChar{\PLSMlistelegv}{\j}} ;}
			}{}
	}
	{
	\end{tikzpicture}
	}

\NewDocumentCommand\SudoMaths{ O{} m }{%
	\useKVdefault[PLTIKZSUDOM]
	\setKV[PLTIKZSUDOM]{#1}% on paramètres les nouvelles clés et on les simplifie
	%calculs intermédiaires
	\def\larcolinter{\inteval{\PLSMnbcol/\PLSMnbsubcol}}
	\def\larliginter{\inteval{\PLSMnblig/\PLSMnbsublig}}
	%lecture liste
	\IfEq{#2}{}{}%
	{%
		\setsepchar[.]{§./}%
		\readlist*\SPGrilleSudoMaths{#2}%
	}
	%débt envtik
	\begin{tikzpicture}[x=\PLSMunite,y=\PLSMunite,line join=miter]
		%cases
		\IfEq{#2}{}{}%
		{%
			\foreach \i in {1,2,...,\PLSMnblig}{%
				\foreach \j in {1,2,...,\PLSMnbcol}{%
					\itemtomacro\SPGrilleSudoMaths[\i,\j]\SMcase
					\IfSubStr{\SMcase}{*}%si on veut colorier via *
					{%
						\StrDel{\SMcase}{*}[\SMcaseb]%
						\draw[draw=none,fill=\PLSMcoulcase] ({\j-1},{1-\i}) rectangle++ (1,-1) node[inner sep=0pt,outer sep=0pt,\PLSMcoultexte,font=\PLSMfonte,midway] {\SMcaseb} ;%
					}
					{%
						\draw ({\j-0.5},{0.5-\i}) node[inner sep=0pt,outer sep=0pt,\PLSMcoultexte,font=\PLSMfonte] {\SMcase} ;%
					}
				}
			}%
		}
		%grilles
		\draw[line width=\PLSMepg] (0,0) rectangle ({\PLSMnbcol},{-\PLSMnblig}) ;
		\draw[line width=\PLSMepf,xstep=1,ystep=1] (0,0) grid ({\PLSMnbcol},{-\PLSMnblig}) ;
		\draw[line width=\PLSMepg,xstep=\larcolinter,ystep=\larliginter] (0,0) grid ({\PLSMnbcol},{-\PLSMnblig}) ;
		%légendes
		\ifboolKV[PLTIKZSUDOM]{Legendes}
		{%
			\foreach \i in {1,2,...,\PLSMnbcol}{\draw ({\i-0.5},{\PLSMdecalleg}) node[inner sep=0pt,outer sep=0pt,font=\PLSMfonteleg] {\strut\StrChar{\PLSMlistelegh}{\i}} ;}
			\foreach \j in {1,2,...,\PLSMnblig}{\draw ({-\PLSMdecalleg},{0.5-\j}) node[inner sep=0pt,outer sep=0pt,font=\PLSMfonteleg] {\StrChar{\PLSMlistelegv}{\j}} ;}
		}{}
	\end{tikzpicture}
}

%=====FRACTALES
\usetikzlibrary{lindenmayersystems}

\pgfdeclarelindenmayersystem{FloconKoch}{
	\rule{F -> F-F++F-F}}
\pgfdeclarelindenmayersystem{TriangleSierpinski}{
	\rule{F -> G-F-G}
	\rule{G -> F+G+F}}
	
\pgfdeclarelindenmayersystem{SierpinskiTriangle}{
    \symbol{X}{\pgflsystemdrawforward}
    \symbol{Y}{\pgflsystemdrawforward}
    \rule{X -> X-Y+X+Y-X}
    \rule{Y -> YY}
}%

\defKV[tikzfract]{%
	Epaisseur=\def\fracttikzthick{#1},%
	Type=\def\fracttikztype{#1},%
	Couleur=\def\fracttikzcolor{#1},%
	LongueurCote=\def\fracttikzlg{#1},%
	Etape=\def\fracttikzstep{#1},%
	Remplissage=\def\fracttikzfill{#1},%
	Depart=\def\fracttikzdepart{#1},%
	Offset=\def\fracttikzoffset{#1}
}

\setKVdefault[tikzfract]{%
	Epaisseur=0.6pt,%
	Type=Koch,%
	Couleur=black,%
	LongueurCote=3,%
	Etape=1,%
	Remplir=false,%
	Remplissage=lightgray,%
	Depart={(0,0)},%
	AlignV=false,%
	Offset=2pt
}

\NewDocumentCommand\FractaleTikz{ s O{} D<>{} }{%
	\restoreKV[tikzfract]%
	\setKV[tikzfract]{#2}%
	\def\fracttikzlgstep{\fpeval{(\fracttikzlg)/(3^\fracttikzstep)}}%
	\IfStrEq{\fracttikztype}{Sierp}%
		{%
			\def\fracttikzlgstep{\fpeval{(\fracttikzlg)/(2^\fracttikzstep)}}%
		}{}%
	\IfBooleanF{#1}{\begin{tikzpicture}[#3]}
	\ifboolKV[tikzfract]{Remplir}%
		{%
			\IfStrEq{\fracttikztype}{Koch}%
				{%
					\ifboolKV[tikzfract]{AlignV}%
						{%
							\node at ($(1/3*\fracttikzlg,0)+(-60:1/3*\fracttikzlg)$) {} ;
						}{}%
					\draw[line width =\fracttikzthick,shift=\fracttikzdepart,draw=\fracttikzcolor,fill=\fracttikzfill,l-system={FloconKoch,step=\fracttikzlgstep cm,angle=60,axiom=F++F++F,order=\fracttikzstep}] lindenmayer system -- cycle;
				}{}%
			\IfStrEq{\fracttikztype}{Sierp}%
				{%
					\fill[\fracttikzcolor] (0,0) -- ++(0:\fracttikzlg cm) -- ++(120:\fracttikzlg cm) -- cycle;
					\draw[draw=none,shift=\fracttikzdepart,fill=white,l-system={SierpinskiTriangle,step=\fracttikzlgstep cm,angle=-120,axiom=X,order=\fracttikzstep}] lindenmayer system -- cycle;
				}{}%
		}%
		{%
			\IfStrEq{\fracttikztype}{Koch}%
				{%
					\ifboolKV[tikzfract]{AlignV}%
						{%
							\node at ($(1/3*\fracttikzlg,0)+(-60:1/3*\fracttikzlg)$) {} ;
						}{}%
					\draw [line width=\fracttikzthick,shift=\fracttikzdepart,\fracttikzcolor,l-system={FloconKoch,step=\fracttikzlgstep cm,angle=60,axiom=F++F++F,order=\fracttikzstep}] lindenmayer system -- cycle;
				}{}%
			\IfStrEq{\fracttikztype}{Sierp}%
				{%
					\fill[\fracttikzcolor] (0,0) -- ++(0:\fracttikzlg cm) -- ++(120:\fracttikzlg cm) -- cycle;
					\draw[draw=none,shift=\fracttikzdepart,fill=white,l-system={SierpinskiTriangle,step=\fracttikzlgstep cm,angle=-120,axiom=X,order=\fracttikzstep}] lindenmayer system -- cycle;
				}{}%
		}%
	\IfBooleanF{#1}{\end{tikzpicture}}%
}

\NewDocumentCommand\EtapesFloconKoch{ O{} D<>{} m }{%
	\restoreKV[tikzfract]%
	\setKV[tikzfract]{#1}%
	\foreach \i in {#3} {%
		\FractaleTikz[AlignV,#1,Etape=\i]<#2>\hspace{\fracttikzoffset}%
	}
}

\NewDocumentCommand\EtapesTapisSierpinski{ O{} D<>{} m }{%
	\restoreKV[tikzfract]%
	\setKV[tikzfract]{#1}%
	\foreach \i in {#3} {%
		\FractaleTikz[Type=Sierp,#1,Etape=\i]<#2>\hspace{\fracttikzoffset}%
	}
}

\endinput