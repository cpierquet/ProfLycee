% proflycee-tools-ecritures.tex
% Copyright 2023  Cédric Pierquet
% Released under the LaTeX Project Public License v1.3c or later, see http://www.latex-project.org/lppl.txt

%===PACKAGE
\RequirePackage{interval}
\RequirePackage{esvect}
\RequirePackage{ifthen}
\RequirePackage{xspace}
\RequirePackage{mleftright} %pour une meilleure gestion des espacements avec \left \right :-)
%\RequirePackage{mathrsfs}%pour \mathscr (à voir...)

%===ENSEMBLES CLASSIQUES
\ifthenelse{\isundefined{\N}}%
	{%
		\NewDocumentCommand\N{ s }{\IfBooleanTF{#1}{\ensuremath{\mathbb{N}^{*}}\xspace}{\ensuremath{\mathbb{N}}\xspace}}%
	}%
	{%
		\RenewDocumentCommand\N{ s }{\IfBooleanTF{#1}{\ensuremath{\mathbb{N}^{*}}\xspace}{\ensuremath{\mathbb{N}}\xspace}}%
	}%
\ifthenelse{\isundefined{\Z}}%
	{%
		\NewDocumentCommand\Z{ s }{\IfBooleanTF{#1}{\ensuremath{\mathbb{Z}^{*}}\xspace}{\ensuremath{\mathbb{Z}}\xspace}}%
	}%
	{%
		\RenewDocumentCommand\Z{ s }{\IfBooleanTF{#1}{\ensuremath{\mathbb{Z}^{*}}\xspace}{\ensuremath{\mathbb{Z}}\xspace}}%
	}%
\ifthenelse{\isundefined{\D}}%
	{%
		\NewDocumentCommand\D{ s }{\IfBooleanTF{#1}{\ensuremath{\mathbb{D}^{*}}\xspace}{\ensuremath{\mathbb{D}}\xspace}}%
	}%
	{%
		\RenewDocumentCommand\D{ s }{\IfBooleanTF{#1}{\ensuremath{\mathbb{D}^{*}}\xspace}{\ensuremath{\mathbb{D}}\xspace}}%
	}%
\ifthenelse{\isundefined{\Q}}%
	{%
		\NewDocumentCommand\Q{ s }{\IfBooleanTF{#1}{\ensuremath{\mathbb{Q}^{*}}\xspace}{\ensuremath{\mathbb{Q}}\xspace}}%
	}%
	{%
		\RenewDocumentCommand\Q{ s }{\IfBooleanTF{#1}{\ensuremath{\mathbb{Q}^{*}}\xspace}{\ensuremath{\mathbb{Q}}\xspace}}%
	}%
\ifthenelse{\isundefined{\R}}%
	{%
		\NewDocumentCommand\R{ s }{\IfBooleanTF{#1}{\ensuremath{\mathbb{R}^{*}}\xspace}{\ensuremath{\mathbb{R}}\xspace}}%
	}%
	{%
		\RenewDocumentCommand\R{ s }{\IfBooleanTF{#1}{\ensuremath{\mathbb{R}^{*}}\xspace}{\ensuremath{\mathbb{R}}\xspace}}%
	}%
\ifthenelse{\isundefined{\C}}%
	{%
		\NewDocumentCommand\C{ s }{\IfBooleanTF{#1}{\ensuremath{\mathbb{C}^{*}}\xspace}{\ensuremath{\mathbb{C}}\xspace}}%
	}%
	{%
		\RenewDocumentCommand\C{ s }{\IfBooleanTF{#1}{\ensuremath{\mathbb{C}^{*}}\xspace}{\ensuremath{\mathbb{C}}\xspace}}%
	}%
\ifthenelse{\isundefined{\ensH}}%
	{%
		\NewDocumentCommand\ensH{ s }{\IfBooleanTF{#1}{\ensuremath{\mathbb{H}^{*}}\xspace}{\ensuremath{\mathbb{H}}\xspace}}%
	}%
	{%
		\RenewDocumentCommand\ensH{ s }{\IfBooleanTF{#1}{\ensuremath{\mathbb{H}^{*}}\xspace}{\ensuremath{\mathbb{H}}\xspace}}%
	}%

%====INTERVALLES
\intervalconfig{separator symbol=;}
\NewDocumentCommand\IntervalleFF{ O{scaled} m m }{\ensuremath{\interval[#1]{#2}{#3}}}
\NewDocumentCommand\IntervalleOF{ O{scaled} m m }{\ensuremath{\interval[#1,open left]{#2}{#3}}}
\NewDocumentCommand\IntervalleFO{ O{scaled} m m }{\ensuremath{\interval[#1,open right]{#2}{#3}}}
\NewDocumentCommand\IntervalleOO{ O{scaled} m m }{\ensuremath{\interval[#1,open]{#2}{#3}}}

%====ARRONDI
\DeclareDocumentCommand\Arrondi{ s O{3} m }{% * pour afficher signe / opt = précision / argument = nb
	\IfBooleanTF{#1}{\num[print-implicit-plus]{\xinteval{round(#3,#2)}}}{\num{\xinteval{round(#3,#2)}}}%
}

%====MODULO
\NewDocumentCommand\Modulo{ s O{Cro} m }{%
	\IfStrEq{#2}{Cro}%
		{\IfBooleanTF{#1}{\quad}{\:\:}[#3]}{}%
	\IfStrEq{#2}{Par}%
		{\IfBooleanTF{#1}{\quad}{\:\:}(#3)}{}%
	\IfStrEq{#2}{Txt}%
		{\IfBooleanTF{#1}{\:\:}{\:}\text{modulo }#3}{}%
}

%====COURBE
\NewDocumentCommand\Courbe{ s O{} }{%
	\IfBooleanTF{#1}%
	{%
		\IfNoValueTF{#2}%
			{\ensuremath{{\mathscr{C}}}\xspace}%
			{\ensuremath{{\mathscr{C}}_{#2}}\xspace}%
	}%
	{%
		\IfNoValueTF{#2}%
			{\ensuremath{{\mathcal{C}}}\xspace}%
			{\ensuremath{{\mathcal{C}}_{#2}}\xspace}%
	}%
}

%====SUITE
\NewDocumentCommand\Suite{ O{n} m }{%
	\ensuremath{\left( #2_{#1} \right)}%
}

%===DIVERS
\AtBeginDocument{%voir hyperref...
	\ifthenelse{\isundefined{\i}}{\newcommand\i{{\rm i}}}{\renewcommand\i{{\rm i}}}%
	\ifthenelse{\isundefined{\e}}{\newcommand\e{{\rm e}}}{\renewcommand\e{{\rm e}}}%
	\ifthenelse{\isundefined{\j}}{\newcommand\j{{\rm j}}}{\renewcommand\j{{\rm j}}}%
}

\ifthenelse{\isundefined{\jfexp}}%
	{\newcommand\jfexp{\ensuremath{\e^{\i\frac{\pi}{3}}}}}%
	{\renewcommand\jfexp{\ensuremath{\e^{\i\frac{\pi}{3}}}}}%

\ifthenelse{\isundefined{\jfalg}}%
	{\newcommand\jfalg{\ensuremath{\frac{1}{2}+\i\frac{\sqrt{3}}{2}}}}%
	{\renewcommand\jfalg{\ensuremath{\frac{1}{2}+\i\frac{\sqrt{3}}{2}}}}%

\ifthenelse{\isundefined{\Esper}}%
	{\newcommand\Esper[2][\mathbb{E}]{\ensuremath{{#1}{\left({#2}\right)}}}}%
	{\renewcommand\Esper[2][\mathbb{E}]{\ensuremath{{#1}{\left({#2}\right)}}}}%

\ifthenelse{\isundefined{\Varianc}}%
	{\newcommand\Varianc[2][\mathbb{V}]{\ensuremath{{#1}{\left({#2}\right)}}}}%
	{\renewcommand\Varianc[2][\mathbb{V}]{\ensuremath{{#1}{\left({#2}\right)}}}}%

\ifthenelse{\isundefined{\EcType}}%
	{\newcommand\EcType[1]{\ensuremath{\sigma{\left({#1}\right)}}}}%
	{\renewcommand\EcType[1]{\ensuremath{\sigma{\left({#1}\right)}}}}%

\ifthenelse{\isundefined{\dx}}%
	{\newcommand\dx[1][x]{\ensuremath{~\text{d}#1}}}%
	{\renewcommand\dx[1][x]{\ensuremath{~\text{d}#1}}}%

\ifthenelse{\isundefined{\Integrale}}%
	{\newcommand\Integrale{\displaystyle\int}}%
	{\renewcommand\Integrale{\displaystyle\int}}%

\ifthenelse{\isundefined{\Limite}}%
	{%
		\NewDocumentCommand\Limite{ s m m o }{%
			\IfBooleanTF{#1}{\ensuremath{\lim_{\IfNoValueTF{#4}{#3}{\substack{#3 \\ #4}}} #2}}{\ensuremath{\lim\limits_{\IfNoValueTF{#4}{#3}{\substack{#3 \\ #4}}} #2}}}\xspace%
	}%
	{%
		\RenewDocumentCommand\Limite{ s m m o }{%
			\IfBooleanTF{#1}{\ensuremath{\lim_{\IfNoValueTF{#4}{#3}{\substack{#3 \\ #4}}} #2}}{\ensuremath{\lim\limits_{\IfNoValueTF{#4}{#3}{\substack{#3 \\ #4}}} #2}}}\xspace%
	}%

%====VALEUR ABSOLUE, NORME
\DeclarePairedDelimiter\ValAbs{\lvert}{\rvert}%
\DeclarePairedDelimiter\ModuleCplx{\lvert}{\rvert}%
\DeclarePairedDelimiter\Norme{\lVert}{\rVert}%
\makeatletter
\let\TmpValAbs\ValAbs
\let\TmpModule\ModuleCplx
\let\TmpNorme\Norme
\def\ValAbs{\@ifstar{\TmpValAbs}{\TmpValAbs*}}
\def\ModuleCplx{\@ifstar{\TmpModule}{\TmpModule*}}
\def\Norme{\@ifstar{\TmpNorme}{\TmpNorme*}}
\makeatother

%====PROBAS
\NewDocumentCommand\LoiNormale{ s m m }{%
	\IfBooleanTF{#1}{\ensuremath{\mathscr{N}{\left(#2;#3\right)}}}{\ensuremath{\mathcal{N}{\left(#2;#3\right)}}}%
}
\NewDocumentCommand\LoiBinomiale{ s m m }{%
	\IfBooleanTF{#1}{\ensuremath{\mathscr{B}{\left(#2;#3\right)}}}{\ensuremath{\mathcal{B}{\left(#2;#3\right)}}}%
}
\NewDocumentCommand\LoiPoisson{ s m }{%
	\IfBooleanTF{#1}{\ensuremath{\mathscr{P}_{#2}}}{\ensuremath{\mathcal{P}_{#2}}}%
}
\NewDocumentCommand\LoiUnif{ s m }{%
	\IfBooleanTF{#1}{\ensuremath{\mathscr{U}_{#2}}}{\ensuremath{\mathcal{U}_{#2}}}%
}
\NewDocumentCommand\LoiExpo{ s m }{%
	\IfBooleanTF{#1}{\ensuremath{\mathscr{E}_{#2}}}{\ensuremath{\mathcal{E}_{#2}}}%
}

%====COORDONNEES, VECTEURS
\NewDocumentCommand\CoordPtPl{ m m }{\ensuremath{\left(#1;#2\right)}}
\NewDocumentCommand\CoordPtEsp{ m m m }{\ensuremath{\left(#1;#2;#3\right)}}
\NewDocumentCommand\CoordVecPl{ m m }{\ensuremath{\begin{pmatrix} #1 \\ #2 \end{pmatrix}}}
\NewDocumentCommand\CoordVecEsp{ m m m }{\ensuremath{\begin{pmatrix} #1 \\ #2 \\ #3 \end{pmatrix}}}
\NewDocumentCommand\MatDeux{ m m m m }{\ensuremath{\begin{pmatrix} #1 & #2 \\ #3 & #4 \end{pmatrix}}}
\NewDocumentCommand\Vecteur{ s m O{} }{%
	\IfBooleanTF{#1}{\ensuremath{\vv*{#2}{#3}}}{\ensuremath{\vv{#2}}}%
}

%====REPÈRES
\setKVdefault[ecrituresreperes]{%
	Sep={;}
}

\NewDocumentCommand\RepereOij{ s O{} }{%
	\useKVdefault[ecrituresreperes]%
	\setKV[ecrituresreperes]{#2}%
	\IfBooleanTF{#1}%si étoilé := on n'aligne pas les flèches + nosmash
		{%
			\ensuremath{\left(O\useKV[ecrituresreperes]{Sep}\Vecteur{\imath},\Vecteur{\jmath}\right)}\xspace%
		}%
		{%
			\ensuremath{\left(O\useKV[ecrituresreperes]{Sep}\Vecteur{\vphantom{t}\imath},\Vecteur{\vphantom{t}\jmath}\right)}\xspace%
		}%
}

\NewDocumentCommand\RepereOuv{ s O{} }{%
	\useKVdefault[ecrituresreperes]%
	\setKV[ecrituresreperes]{#2}%
	\IfBooleanTF{#1}%si étoilé := on n'aligne pas les flèches + nosmash
		{%
			\ensuremath{\left(O\useKV[ecrituresreperes]{Sep}\Vecteur{u},\Vecteur{v}\right)}\xspace%
		}%
		{%
			\ensuremath{\left(O\useKV[ecrituresreperes]{Sep}\Vecteur{\vphantom{t}u},\Vecteur{\vphantom{t}v}\right)}\xspace%
		}%
}

\NewDocumentCommand\RepereOijk{ s O{} }{%
	\useKVdefault[ecrituresreperes]%
	\setKV[ecrituresreperes]{#2}%
	\IfBooleanTF{#1}%si étoilé := on n'aligne pas les flèches + nosmash
		{%
			\ensuremath{\left(O\useKV[ecrituresreperes]{Sep}\Vecteur{\imath},\Vecteur{\jmath},\Vecteur{k}\right)}\xspace%
		}%
		{%
			\ensuremath{\left(O\useKV[ecrituresreperes]{Sep}\Vecteur{\vphantom{t}\imath},\Vecteur{\vphantom{t}\jmath},\Vecteur{\vphantom{t}\smash{k}}\right)}\xspace%
		}%
}

\NewDocumentCommand\ReperePlan{ s O{} m m m }{%
	\useKVdefault[ecrituresreperes]%
	\setKV[ecrituresreperes]{#2}%
	\IfBooleanTF{#1}%si étoilé := on n'aligne pas les flèches + nosmash
		{%
			\def\vecteurun{#4}\def\vecteurdeux{#5}%
		}%
		{%
			\def\vecteurun{\smash{#4}}\def\vecteurdeux{\smash{#5}}%
		}%
	\IfStrEq{#4}{i}{\def\vecteurun{\imath}}{}%
	\IfStrEq{#4}{j}{\def\vecteurun{\jmath}}{}%
	\IfStrEq{#5}{i}{\def\vecteurdeux{\imath}}{}%
	\IfStrEq{#5}{j}{\def\vecteurdeux{\jmath}}{}%
	\IfBooleanTF{#1}%si étoilé := on n'aligne pas les flèches + nosmash
		{%
			\ensuremath{\left(#3\useKV[ecrituresreperes]{Sep}\Vecteur{\vecteurun},\Vecteur{\vecteurdeux}\right)}%
		}%
		{%
			\ensuremath{\left(#3\useKV[ecrituresreperes]{Sep}\Vecteur{\vphantom{t}\vecteurun},\Vecteur{\vphantom{t}\vecteurdeux}\right)}%
		}%
}

\NewDocumentCommand\RepereEspace{ s O{} m m m m }{%
	\useKVdefault[ecrituresreperes]%
	\setKV[ecrituresreperes]{#2}%
	\IfBooleanTF{#1}%si étoilé := on n'aligne pas les flèches + nosmash
		{%
			\def\vecteurun{#4}\def\vecteurdeux{#5}\def\vecteurtrois{#6}%
		}%
		{%
			\def\vecteurun{\smash{#4}}\def\vecteurdeux{\smash{#5}}\def\vecteurtrois{\smash{#6}}%
		}%
	\IfStrEq{#4}{i}{\def\vecteurun{\imath}}{}%
	\IfStrEq{#4}{j}{\def\vecteurun{\jmath}}{}%
	\IfStrEq{#5}{i}{\def\vecteurdeux{\imath}}{}%
	\IfStrEq{#5}{j}{\def\vecteurdeux{\jmath}}{}%
	\IfStrEq{#6}{i}{\def\vecteurtrois{\imath}}{}%
	\IfStrEq{#6}{j}{\def\vecteurtrois{\jmath}}{}%
	\IfBooleanTF{#1}%si étoilé := on n'aligne pas les flèches + nosmash
		{%
			\ensuremath{\left(#3\useKV[ecrituresreperes]{Sep}\Vecteur{\vecteurun},\Vecteur{\vecteurdeux},\Vecteur{\vecteurtrois}\right)}%
		}%
		{%
			\ensuremath{\left(#3\useKV[ecrituresreperes]{Sep}\Vecteur{\vphantom{t}\vecteurun},\Vecteur{\vphantom{t}\vecteurdeux},\Vecteur{\vphantom{t}\vecteurtrois}\right)}%
		}%
}

%====PROBAS
\setKVdefault[probascondit]{%
	min=false,%
	Maj=false,%
	BB=false,%
	Formule=false
}

\newcommand\notationproba{P}%

\NewDocumentCommand\ProbaCondit{ O{} m m }{%
	\useKVdefault[probascondit]%
	\setKV[probascondit]{#1}%
	%gestion des booléens
	\ifboolKV[probascondit]{Maj}{\def\notationproba{P}}{}%
	\ifboolKV[probascondit]{min}{\def\notationproba{p}}{}%
	\ifboolKV[probascondit]{BB}{\def\notationproba{\mathbb{P}}}{}%
	\notationproba_{#3}{\left(#2\right)} \ifboolKV[probascondit]{Formule}{= \dfrac{\notationproba{\left(#2 \cap #3\right)}}{\notationproba{\left(#3\right)}}}{}%
}

%====IF, IC
\defKV[IntFluc]{%
	Seuil=\def\intflucconfProba{#1},%
	Classe=\def\intflucconfNiv{#1},%
	p=\def\intflucconfP{#1},%
	n=\def\intflucconfN{#1},%
	Arrondi=\def\intflucconfPrec{#1}
}
\setKVdefault[IntFluc]{%
	Seuil=95,%
	Classe=Term,%
	Arrondi=2,%
	Calcul=false,%
	Details=false,%
	Formule=true,%
	p={},%
	n={},%
	Symbole=\approx
}

\NewDocumentCommand\IntFluctu{ O{} }{%
	\useKVdefault[IntFluc]%
	\setKV[IntFluc]{#1}%
	\IfStrEqCase{\intflucconfProba}{%
		{90}{\def\ualpha{1.64}}%
		{95}{\def\ualpha{1.96}}%
		{99}{\def\ualpha{2.58}}%
		%{99.9}{\def\ualpha{3.29}}%
	}%
	\ifboolKV[IntFluc]{Formule}%
	{%
		\IfStrEqCase{\intflucconfNiv}{%
			{2de}{\IntervalleFF{ p - \dfrac{1}{\sqrt{n}} }{ p + \dfrac{1}{\sqrt{n}}}}%
			{Term}{\IntervalleFF{ p-\num{\ualpha}\dfrac{\sqrt{p(1-p)}}{\sqrt{n}} }{ p+\num{\ualpha}\dfrac{\sqrt{p(1-p)}}{\sqrt{n}} }}%
		}%
	}%
	{}%
	\ifboolKV[IntFluc]{Details}%
	{%
		\IfSubStr{\intflucconfP}{/}%
		{%
			\def\ValIFp{\ConversionFraction[t]{\intflucconfP}}\def\ValIFpD{\ConversionFraction[d]{\intflucconfP}}%
		}%
		{%
			\def\ValIFp{\num{\intflucconfP}}\def\ValIFpD{\num{\intflucconfP}}%
		}%
		=
		\IfStrEqCase{\intflucconfNiv}{%
			{2de}{\IntervalleFF{ \ValIFpD - \dfrac{1}{\sqrt{\num{\intflucconfN}}} }{ \ValIFpD + \dfrac{1}{\sqrt{\num{\intflucconfN}}}}}%
			{Term}{\IntervalleFF{ \ValIFpD-\num{\ualpha}\dfrac{\sqrt{\ValIFp\mleft(1-\ValIFp\mright)}}{\sqrt{\num{\intflucconfN}}} }{ \ValIFpD+\num{\ualpha}\dfrac{\sqrt{\ValIFp\mleft(1-\ValIFp\mright)}}{\sqrt{\num{\intflucconfN}}} }}%
		}%
	}%
	{}%
	\ifboolKV[IntFluc]{Calcul}%
	{%
		\IfStrEqCase{\intflucconfNiv}{%calculs avec xfp ??
			{2de}%
			{%
				\def\calcborneinfIF{ \fpeval{ max(floor(\intflucconfP - 1/sqrt(\intflucconfN),\intflucconfPrec),0) } }%
				\def\calcbornesupIF{ \fpeval{ min(ceil(\intflucconfP + 1/sqrt(\intflucconfN),\intflucconfPrec),1) } }%
			}%
			{Term}%
			{%
				\def\calcborneinfIF{ \fpeval{ max(floor(\intflucconfP - \ualpha*sqrt(\intflucconfP*(1-\intflucconfP))/sqrt(\intflucconfN),\intflucconfPrec),0) } }%
				\def\calcbornesupIF{ \fpeval{ min(ceil(\intflucconfP + \ualpha*sqrt(\intflucconfP*(1-\intflucconfP))/sqrt(\intflucconfN),\intflucconfPrec),1) } }%
			}%
		}%
		\useKV[IntFluc]{Symbole} \IntervalleFF{ \num{\calcborneinfIF} }{ \num{\calcbornesupIF} }%
	}%
	{}%
}

\NewDocumentCommand\RedactionIntFluct{ O{} m }{%
	%\setKV[IntFluc]{#1}%
	\begin{flalign*}
		#2 &= \IntFluctu[Formule,Details=false,Calcul=false] && \\
		   &  \IntFluctu[Formule=false,Details=true,Calcul=false,#1] && \\
		   &  \IntFluctu[Formule=false,Details=false,Calcul=true,#1] && \\
	\end{flalign*}%
}

\defKV[IntConf]{%
	Seuil=\def\intflucconfProba{#1},%
	Classe=\def\intflucconfNiv{#1},%
	f=\def\intflucconfF{#1},%
	n=\def\intflucconfN{#1},%
	Arrondi=\def\intflucconfPrec{#1}
}
\setKVdefault[IntConf]{%
	Seuil=95,%
	Classe=Term,%
	Arrondi=2,%
	Calcul=false,%
	Details=false,%
	Formule=true,%
	f={},%
	n={},%
	Symbole=\approx
}

\NewDocumentCommand\IntConf{ O{} }{%
	\useKVdefault[IntConf]%
	\setKV[IntConf]{#1}%
	\IfStrEqCase{\intflucconfProba}{%
		{90}{\def\ualpha{1.64}}%
		{95}{\def\ualpha{1.96}}%
		{99}{\def\ualpha{2.58}}%
		%{99.9}{\def\ualpha{3.29}}%
	}%
	\ifboolKV[IntConf]{Formule}%
	{%
		\IfStrEqCase{\intflucconfNiv}{%
			{2de}{\IntervalleFF{ f - \dfrac{1}{\sqrt{n}} }{ f + \dfrac{1}{\sqrt{n}}}}%
			{Term}{\IntervalleFF{ f-\num{\ualpha}\dfrac{\sqrt{f(1-f)}}{\sqrt{n}} }{ f+\num{\ualpha}\dfrac{\sqrt{f(1-f)}}{\sqrt{n}} }}%
		}%
	}%
	{}%
	\ifboolKV[IntConf]{Details}%
	{%
		\IfSubStr{\intflucconfF}{/}%
		{%
			\def\ValIFp{\ConversionFraction[t]{\intflucconfF}}\def\ValIFpD{\ConversionFraction[d]{\intflucconfF}}%
		}%
		{%
			\def\ValIFp{\num{\intflucconfF}}\def\ValIFpD{\num{\intflucconfF}}%
		}%
		=
		\IfStrEqCase{\intflucconfNiv}{%
			{2de}{\IntervalleFF{ \ValIFpD - \dfrac{1}{\sqrt{\num{\intflucconfN}}} }{ \ValIFpD + \dfrac{1}{\sqrt{\num{\intflucconfN}}}}}%
			{Term}{\IntervalleFF{ \ValIFpD-\num{\ualpha}\dfrac{\sqrt{\ValIFp\mleft(1-\ValIFp\mright)}}{\sqrt{\num{\intflucconfN}}} }{ \ValIFpD+\num{\ualpha}\dfrac{\sqrt{\ValIFp\mleft(1-\ValIFp\mright)}}{\sqrt{\num{\intflucconfN}}} }}%
		}%
	}%
	{}%
	\ifboolKV[IntConf]{Calcul}%
	{%
		\IfStrEqCase{\intflucconfNiv}{%calculs avec xfp ??
			{2de}%
			{%
				\def\calcborneinfIF{ \fpeval{ max(floor(\intflucconfF - 1/sqrt(\intflucconfN),\intflucconfPrec),0) } }%
				\def\calcbornesupIF{ \fpeval{ min(ceil(\intflucconfF + 1/sqrt(\intflucconfN),\intflucconfPrec),1) } }%
			}%
			{Term}%
			{%
				\def\calcborneinfIF{ \fpeval{ max(floor(\intflucconfF - \ualpha*sqrt(\intflucconfF*(1-\intflucconfF))/sqrt(\intflucconfN),\intflucconfPrec),0) } }%
				\def\calcbornesupIF{ \fpeval{ min(ceil(\intflucconfF + \ualpha*sqrt(\intflucconfF*(1-\intflucconfF))/sqrt(\intflucconfN),\intflucconfPrec),1) } }%
			}%
		}%
		\useKV[IntConf]{Symbole}
		\IntervalleFF{ \num{\calcborneinfIF} }{ \num{\calcbornesupIF} }%
	}%
	{}%
}

\NewDocumentCommand\RedactionIntConf{ O{} m }{%
	%\setKV[IntConf]{#1}%
	\begin{flalign*}
		#2 &= \IntConf[Formule,Details=false,Calcul=false] && \\
		   &  \IntConf[Formule=false,Details=true,Calcul=false,#1] && \\
		   &  \IntConf[Formule=false,Details=false,Calcul=true,#1] && \\
	\end{flalign*}%
}

\endinput