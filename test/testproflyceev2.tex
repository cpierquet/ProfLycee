% !TeX TXS-program:compile = txs:///pythonpdfse

\documentclass[a4paper, french]{article}
\usepackage[french]{babel}
\usepackage[utf8]{inputenc}
\usepackage[T1]{fontenc}
\usepackage{geometry}
\geometry{margin=1.5cm}
\usepackage{ProfLycee}
\usepackage{ProfCollege}
\usepackage{minted}

\begin{document}

\part*{Utilisation basique du package \texttt{ProfLycee}}

\section{Préambule de \og test \fg}

\begin{minted}[frame=lines,framesep=2mm,bgcolor=lightgray!5,fontsize=\footnotesize,tabsize=4]{tex}
%préambule
\documentclass[a4paper, french]{article}
\usepackage[utf8]{inputenc}
\usepackage{geometry}
\geometry{margin=1.5cm}
\usepackage{ProfCollege}
\usepackage{ProfLycee}

\begin{document}

...

\end{document}
\end{minted}

\section{Outils \og splinetikz \fg{} et \og tangentetikz \fg}

\begin{minted}[frame=lines,framesep=2mm,bgcolor=lightgray!5,fontsize=\footnotesize,tabsize=4]{tex}
\begin{center}
	\begin{tikzpicture}[x=1.25cm,y=1.25cm]
		\draw[xstep=0.5,ystep=0.5,line width=0.3pt,lightgray!50] (-1,-1) grid (11,5);
		\draw[xstep=1,ystep=1,line width=0.6pt,lightgray!50] (-1,-1) grid (11,5) ;
		\draw[line width=1.5pt,->,gray] (-1,0)--(11,0) ;
		\draw[line width=1.5pt,->,gray] (0,-1)--(0,5) ;
		\foreach \x in {0,1,...,10} {\draw[gray,line width=1.5pt] (\x,4pt) -- (\x,-4pt) ;}
		\foreach \y in {0,1,...,4} {\draw[gray,line width=1.5pt] (4pt,\y) -- (-4pt,\y) ;}
		\draw[darkgray] (1,-4pt) node[below,font=\sffamily] {1} ;
		\draw[darkgray] (-4pt,1) node[left,font=\sffamily] {1} ;
		%liste des points de cotrôle
		\def\LISTE{0/1/0§4/3.667/-0.333§7.5/1.75/0§9/2/-0.333§10/0/-10}
		\splinetikz[liste=\LISTE,affpoints=true,coeffs=3,couleur=red]
		\tangentetikz[liste=\LISTE,xl=0,xr=1,couleur=ForestGreen,style=dashed]
		\tangentetikz[liste=\LISTE,xl=1,xr=1,couleur=ForestGreen,style=dashed,point=2]
		\tangentetikz[liste=\LISTE,xl=1,xr=1,couleur=ForestGreen,style=dashed,point=3]
		\tangentetikz[liste=\LISTE,xl=1,xr=1,couleur=ForestGreen,style=dashed,point=4]
		\tangentetikz[liste=\LISTE,xl=0.5,xr=0,couleur=ForestGreen,style=dashed,point=5]
	\end{tikzpicture}
\end{center}
\end{minted}

\begin{center}
	\begin{tikzpicture}[x=1.25cm,y=1.25cm]
		\draw[xstep=0.5,ystep=0.5,line width=0.3pt,lightgray!50] (-1,-1) grid (11,5);
		\draw[xstep=1,ystep=1,line width=0.6pt,lightgray!50] (-1,-1) grid (11,5) ;
		\draw[line width=1.5pt,->,gray] (-1,0)--(11,0) ;
		\draw[line width=1.5pt,->,gray] (0,-1)--(0,5) ;
		\foreach \x in {0,1,...,10} {\draw[gray,line width=1.5pt] (\x,4pt) -- (\x,-4pt) ;}
		\foreach \y in {0,1,...,4} {\draw[gray,line width=1.5pt] (4pt,\y) -- (-4pt,\y) ;}
		\draw[darkgray] (1,-4pt) node[below,font=\sffamily] {1} ;
		\draw[darkgray] (-4pt,1) node[left,font=\sffamily] {1} ;
		%liste des points de cotrôle
		\def\LISTE{0/1/0§4/3.667/-0.333§7.5/1.75/0§9/2/-0.333§10/0/-10}
		\splinetikz[liste=\LISTE,affpoints=true,coeffs=3,couleur=red]
		\tangentetikz[liste=\LISTE,xl=0,xr=1,couleur=ForestGreen,style=dashed]
		\tangentetikz[liste=\LISTE,xl=1,xr=1,couleur=ForestGreen,style=dashed,point=2]
		\tangentetikz[liste=\LISTE,xl=1,xr=1,couleur=ForestGreen,style=dashed,point=3]
		\tangentetikz[liste=\LISTE,xl=1,xr=1,couleur=ForestGreen,style=dashed,point=4]
		\tangentetikz[liste=\LISTE,xl=0.5,xr=0,couleur=ForestGreen,style=dashed,point=5]
	\end{tikzpicture}
\end{center}

\section{Outil \og XCas-like \fg}

\begin{minted}[frame=lines,framesep=2mm,bgcolor=lightgray!5,fontsize=\footnotesize,tabsize=4]{tex}
\begin{center}
	\begin{tikzpicture}[x=1cm,y=1cm,line width=1pt]
		\paramCF[titre=true]
		\ligneCF{\textsf{(x+1)\chap2}}{$\mathsf{x^2+2x+1}$}
		\ligneCF{\texttt{(x+1)\chap2}}{$\mathtt{x^2+2x+1}$}
		\ligneCF{\textsf{Dérivée[(x+5)*exp(-0.1*x)]}}{$\mathsf{\rightarrow (0.5-0.1*x)*exp(-0.1*x)}$}
	\end{tikzpicture}
\end{center}
\end{minted}

\begin{center}
	\begin{tikzpicture}[x=1cm,y=1cm,line width=1pt]
		\paramCF[titre=true]
		\ligneCF{\textsf{(x+1)\CFchap2}}{$\mathsf{x^2+2x+1}$}
		\ligneCF{\texttt{(x+1)\CFchap2}}{$\mathtt{x^2+2x+1}$}
		\ligneCF{\textsf{Dérivée[(x+5)*exp(-0.1*x)]}}{$\mathsf{\rightarrow (0.5-0.1*x)*exp(-0.1*x)}$}
	\end{tikzpicture}
\end{center}

\pagebreak

\section{Python}

\subsection{Code Python(tex)}

\begin{minted}[frame=lines,framesep=2mm,bgcolor=lightgray!5,fontsize=\footnotesize,tabsize=4]{tex}
\begin{envcodepythontex}[largeur=12cm]
	#environnement Python(tex) centré avec numéros de ligne
	def f(x) :
		return x**2
\end{envcodepythontex}
\end{minted}

\begin{envcodepythontex}[largeur=12cm]
	#environnement Python(tex) centré avec numéros de ligne
	def f(x) :
		return x**2
\end{envcodepythontex}

\begin{minted}[frame=lines,framesep=2mm,bgcolor=lightgray!5,fontsize=\footnotesize,tabsize=4]{tex}
\begin{envcodepythontex}[largeur=12cm,lignes=false,centre=false]
	#environnement Python(tex) non centré sans numéro de ligne
	def f(x) :
		return x**2
\end{envcodepythontex}
\end{minted}

\begin{envcodepythontex}[largeur=12cm,lignes=false,centre=false]
	#environnement Python(tex) non centré sans numéro de ligne
	def f(x) :
		return x**2
\end{envcodepythontex}

\subsection{Console Python(tex)}

\begin{minted}[frame=lines,framesep=2mm,bgcolor=lightgray!5,fontsize=\footnotesize,tabsize=4]{tex}
\begin{envconsolepythontex}[largeur=14cm,centre=false]
	#console Python(tex) non centrée avec label
	from math import sqrt
	1+1
	sqrt(12)
\end{envconsolepythontex}
\end{minted}

\begin{envconsolepythontex}[largeur=14cm,centre=false]
	#console Python(tex) non centrée avec label
	from math import sqrt
	1+1
	sqrt(12)
\end{envconsolepythontex}

\begin{minted}[frame=lines,framesep=2mm,bgcolor=lightgray!5,fontsize=\footnotesize,tabsize=4]{tex}
\begin{envconsolepythontex}[largeur=14cm,label=false]
	#console Python(tex) centrée sans label
	table = [[1,2],[3,4]]
	table[0][0]
\end{envconsolepythontex}
\end{minted}

\begin{envconsolepythontex}[largeur=14cm,label=false]
	#console Python(tex) centrée sans label
	table = [[1,2],[3,4]]
	table[0][0]
\end{envconsolepythontex}

\begin{minted}[frame=lines,framesep=2mm,bgcolor=lightgray!5,fontsize=\footnotesize,tabsize=4]{tex}
	\begin{envcodepythontex}[largeur=12cm]
		#environnement Python(tex) centré sans numéros de ligne
		def f(x) :
		return x**2
	\end{envcodepythontex}
\end{minted}

\subsection{Code python(minted)}

\begin{minted}[frame=lines,framesep=2mm,bgcolor=lightgray!5,fontsize=\footnotesize,tabsize=4]{tex}
\begin{envcodepythonminted}[12cm][center]
	#environnement Python(minted) centré avec numéros
	def f(x) :
		return x**2
\end{envcodepythonminted}
\end{minted}

\begin{envcodepythonminted}[12cm][center]
def f(x) :
	return x**2
\end{envcodepythonminted}

\begin{minted}[frame=lines,framesep=2mm,bgcolor=lightgray!5,fontsize=\footnotesize,tabsize=4]{tex}
\begin{envcodepythonminted}*[0.8\linewidth][]
	#environnement Python(minted) non centré sans numéros (0.8\linewidth)
	def f(x) :
		return x**2
\end{envcodepythonminted}
\end{minted}

\begin{envcodepythonminted}*[0.8\linewidth][]
	def f(x) :
		return x**2
\end{envcodepythonminted}

\newpage

\section{PseudoCode}

\begin{minted}[frame=lines,framesep=2mm,bgcolor=lightgray!5,fontsize=\footnotesize,tabsize=4]{tex}
\begin{envpseudocode}
%attention à l'indentation, gobble ne fonctionne pas...
List = [...]          # à déclarer au préalable
n = longueur(List)
Pour i allant de 0 à n-1 Faire
	Afficher(List[i])
FinPour
\end{envpseudocode}
\end{minted}

\begin{envpseudocode}
List = [...]          # à déclarer au préalable
n = longueur(List)
Pour i allant de 0 à n-1 Faire
	Afficher(List[i])
FinPour
\end{envpseudocode}

\begin{minted}[frame=lines,framesep=2mm,bgcolor=lightgray!5,fontsize=\footnotesize,tabsize=4]{tex}
\begin{envpseudocode}*
List = [...]          # à déclarer au préalable
n = longueur(List)
Pour i allant de 0 à n-1 Faire
	Afficher(List[i])
FinPour
\end{envpseudocode}
\end{minted}

\begin{envpseudocode}*
List = [...]          # à déclarer au préalable
n = longueur(List)
Pour i allant de 0 à n-1 Faire
	Afficher(List[i])
FinPour
\end{envpseudocode}

\begin{minted}[frame=lines,framesep=2mm,bgcolor=lightgray!5,fontsize=\footnotesize,tabsize=4]{tex}
\begin{envpseudocode}*[15cm][center]
List = [...]          # à déclarer au préalable
n = longueur(List)
Pour i allant de 0 à n-1 Faire
	Afficher(List[i])
FinPour
\end{envpseudocode}
\end{minted}

\begin{envpseudocode}*[15cm][center]
List = [...]          # à déclarer au préalable
n = longueur(List)
Pour i allant de 0 à n-1 Faire
	Afficher(List[i])
FinPour
\end{envpseudocode}

\newpage

\section{Terminal}

\begin{minted}[frame=lines,framesep=2mm,bgcolor=lightgray!5,fontsize=\footnotesize,tabsize=4]{tex}
\begin{PLtermwin}[15cm]{}
Microsoft Windows [version 10.0.22000.493]
(c) Microsoft Corporation. Tous droits réservés.

C:\Users\test>ping ctan.org

Envoi d'une requête 'ping' sur ctan.org [5.35.249.60] avec 32 octets de données :
Réponse de 5.35.249.60 : octets=32 temps=38 ms TTL=51
Réponse de 5.35.249.60 : octets=32 temps=40 ms TTL=51
\end{PLtermwin}
\end{minted}

\begin{PLtermwin}[15cm]{}
Microsoft Windows [version 10.0.22000.493]
(c) Microsoft Corporation. Tous droits réservés.

C:\Users\test>ping -c 2ctan.org

Envoi d'une requête 'ping' sur ctan.org [5.35.249.60] avec 32 octets de données :
Réponse de 5.35.249.60 : octets=32 temps=38 ms TTL=51
Réponse de 5.35.249.60 : octets=32 temps=40 ms TTL=51
\end{PLtermwin}

\begin{minted}[frame=lines,framesep=2mm,bgcolor=lightgray!5,fontsize=\footnotesize,tabsize=4]{tex}
\begin{PLtermunix}[12cm]{titre=Terminal Ubuntu}
test@DESKTOP:~$ ping -c 2 ctan.org
PING ctan.org (5.35.249.60) 56(84) bytes of data.
64 bytes from comedy.dante.de (5.35.249.60): icmp_seq=1 ttl=51 time=37.7 ms
64 bytes from comedy.dante.de (5.35.249.60): icmp_seq=2 ttl=51 time=37.9 ms
\end{PLtermunix}
\end{minted}

%\begin{PLtermunix}[12cm]{titre=Terminal Ubuntu}
%test@DESKTOP:~$ ping -c 2 ctan.org
%PING ctan.org (5.35.249.60) 56(84) bytes of data.
%64 bytes from comedy.dante.de (5.35.249.60): icmp_seq=1 ttl=51 time=37.7 ms
%64 bytes from comedy.dante.de (5.35.249.60): icmp_seq=2 ttl=51 time=37.9 ms
%\end{PLtermunix}

\begin{minted}[frame=lines,framesep=2mm,bgcolor=lightgray!5,fontsize=\footnotesize,tabsize=4]{tex}
\begin{PLtermosx}[12cm]{titre=Terminal MacOSX}
[test@server]$ ping -c 2 ctan.org
PING ctan.org (5.35.249.60) 56(84) bytes of data.
64 bytes from comedy.dante.de (5.35.249.60): icmp_seq=1 ttl=51 time=37.7 ms
64 bytes from comedy.dante.de (5.35.249.60): icmp_seq=4 ttl=51 time=37.9 ms
\end{PLtermosx}
\end{minted}

\begin{PLtermosx}[12cm]{titre=Terminal MacOSX}
[test@server]$ ping -c 2 ctan.org
PING ctan.org (5.35.249.60) 56(84) bytes of data.
64 bytes from comedy.dante.de (5.35.249.60): icmp_seq=1 ttl=51 time=37.7 ms
64 bytes from comedy.dante.de (5.35.249.60): icmp_seq=4 ttl=51 time=37.9 ms
\end{PLtermosx}

\end{document}