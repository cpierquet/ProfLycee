% !TeX TXS-program:compile = txs:///pythonpdfse

\documentclass[a4paper, french]{article}
\usepackage{geometry}
\geometry{margin=1.5cm}
\usepackage{ProfCollege}
\usepackage{ProfLycee}
\usepackage{minted}
\usepackage[french]{babel}

\begin{document}

\part*{Utilisation basique du package \texttt{ProfLycee}}

\section{Préambule de \og test \fg}

\begin{minted}[frame=lines,framesep=2mm,bgcolor=lightgray!5,fontsize=\footnotesize,tabsize=4]{tex}
%préambule
\documentclass[a4paper, french]{article}
\usepackage[utf8]{inputenc}
\usepackage{geometry}
\geometry{margin=1.5cm}
\usepackage{ProfCollege}
\usepackage{ProfLycee}

\begin{document}

...

\end{document}
\end{minted}

\section{Outils \og splinetikz \fg{} et \og tangentetikz \fg}

\begin{minted}[frame=lines,framesep=2mm,bgcolor=lightgray!5,fontsize=\footnotesize,tabsize=4]{tex}
\begin{center}
	\begin{tikzpicture}[x=1.25cm,y=1.25cm]
		\draw[xstep=0.5,ystep=0.5,line width=0.3pt,lightgray!50] (-1,-1) grid (11,5);
		\draw[xstep=1,ystep=1,line width=0.6pt,lightgray!50] (-1,-1) grid (11,5) ;
		\draw[line width=1.5pt,->,gray] (-1,0)--(11,0) ;
		\draw[line width=1.5pt,->,gray] (0,-1)--(0,5) ;
		\foreach \x in {0,1,...,10} {\draw[gray,line width=1.5pt] (\x,4pt) -- (\x,-4pt) ;}
		\foreach \y in {0,1,...,4} {\draw[gray,line width=1.5pt] (4pt,\y) -- (-4pt,\y) ;}
		\draw[darkgray] (1,-4pt) node[below,font=\sffamily] {1} ;
		\draw[darkgray] (-4pt,1) node[left,font=\sffamily] {1} ;
		%liste des points de cotrôle
		\def\LISTE{0/1/0§4/3.667/-0.333§7.5/1.75/0§9/2/-0.333§10/0/-10}
		\splinetikz[liste=\LISTE,affpoints=true,coeffs=3,couleur=red]
		\tangentetikz[liste=\LISTE,xl=0,xr=1,couleur=ForestGreen,style=dashed]
		\tangentetikz[liste=\LISTE,xl=1,xr=1,couleur=ForestGreen,style=dashed,point=2]
		\tangentetikz[liste=\LISTE,xl=1,xr=1,couleur=ForestGreen,style=dashed,point=3]
		\tangentetikz[liste=\LISTE,xl=1,xr=1,couleur=ForestGreen,style=dashed,point=4]
		\tangentetikz[liste=\LISTE,xl=0.5,xr=0,couleur=ForestGreen,style=dashed,point=5]
	\end{tikzpicture}
\end{center}
\end{minted}

\begin{center}
	\begin{tikzpicture}[x=1.25cm,y=1.25cm]
		\draw[xstep=0.5,ystep=0.5,line width=0.3pt,lightgray!50] (-1,-1) grid (11,5);
		\draw[xstep=1,ystep=1,line width=0.6pt,lightgray!50] (-1,-1) grid (11,5) ;
		\draw[line width=1.5pt,->,gray] (-1,0)--(11,0) ;
		\draw[line width=1.5pt,->,gray] (0,-1)--(0,5) ;
		\foreach \x in {0,1,...,10} {\draw[gray,line width=1.5pt] (\x,4pt) -- (\x,-4pt) ;}
		\foreach \y in {0,1,...,4} {\draw[gray,line width=1.5pt] (4pt,\y) -- (-4pt,\y) ;}
		\draw[darkgray] (1,-4pt) node[below,font=\sffamily] {1} ;
		\draw[darkgray] (-4pt,1) node[left,font=\sffamily] {1} ;
		%liste des points de cotrôle
		\def\LISTE{0/1/0§4/3.667/-0.333§7.5/1.75/0§9/2/-0.333§10/0/-10}
		\splinetikz[liste=\LISTE,affpoints=true,coeffs=3,couleur=red]
		\tangentetikz[liste=\LISTE,xl=0,xr=1,couleur=ForestGreen,style=dashed]
		\tangentetikz[liste=\LISTE,xl=1,xr=1,couleur=ForestGreen,style=dashed,point=2]
		\tangentetikz[liste=\LISTE,xl=1,xr=1,couleur=ForestGreen,style=dashed,point=3]
		\tangentetikz[liste=\LISTE,xl=1,xr=1,couleur=ForestGreen,style=dashed,point=4]
		\tangentetikz[liste=\LISTE,xl=0.5,xr=0,couleur=ForestGreen,style=dashed,point=5]
	\end{tikzpicture}
\end{center}

\section{Outil \og XCas-like \fg}

\begin{minted}[frame=lines,framesep=2mm,bgcolor=lightgray!5,fontsize=\footnotesize,tabsize=4]{tex}
\begin{center}
	\begin{tikzpicture}[x=1cm,y=1cm,line width=1pt]
		\paramCF[titre=true]
		\ligneCF{\textsf{(x+1)\chap2}}{$\mathsf{x^2+2x+1}$}
		\ligneCF{\texttt{(x+1)\chap2}}{$\mathtt{x^2+2x+1}$}
		\ligneCF{\textsf{Dérivée[(x+5)*exp(-0.1*x)]}}{$\mathsf{\rightarrow (0.5-0.1*x)*exp(-0.1*x)}$}
	\end{tikzpicture}
\end{center}
\end{minted}

\begin{center}
	\begin{tikzpicture}[x=1cm,y=1cm,line width=1pt]
		\paramCF[titre=true]
		\ligneCF{\textsf{(x+1)\CFchap2}}{$\mathsf{x^2+2x+1}$}
		\ligneCF{\texttt{(x+1)\CFchap2}}{$\mathtt{x^2+2x+1}$}
		\ligneCF{\textsf{Dérivée[(x+5)*exp(-0.1*x)]}}{$\mathsf{\rightarrow (0.5-0.1*x)*exp(-0.1*x)}$}
	\end{tikzpicture}
\end{center}

\pagebreak

\section{Pythontex}

\subsection{Code Python}

\begin{minted}[frame=lines,framesep=2mm,bgcolor=lightgray!5,fontsize=\footnotesize,tabsize=4]{tex}
\begin{envcodepythontex}[largeur=12cm]
	#environnement Python(tex) centré avec numéros de ligne
	def f(x) :
		return x**2
\end{envcodepythontex}
\end{minted}

\begin{envcodepythontex}[largeur=12cm]
	#environnement Python(tex) centré avec numéros de ligne
	def f(x) :
		return x**2
\end{envcodepythontex}

\begin{minted}[frame=lines,framesep=2mm,bgcolor=lightgray!5,fontsize=\footnotesize,tabsize=4]{tex}
\begin{envcodepythontex}[largeur=12cm,lignes=false,centre=false]
	#environnement Python(tex) non centré sans numéro de ligne
	def f(x) :
		return x**2
\end{envcodepythontex}
\end{minted}

\begin{envcodepythontex}[largeur=12cm,lignes=false,centre=false]
	#environnement Python(tex) non centré sans numéro de ligne
	def f(x) :
		return x**2
\end{envcodepythontex}

\subsection{Console Python}

\begin{minted}[frame=lines,framesep=2mm,bgcolor=lightgray!5,fontsize=\footnotesize,tabsize=4]{tex}
\begin{envconsolepythontex}[largeur=14cm,centre=false]
	#console Python(tex) non centrée avec label
	from math import sqrt
	1+1
	sqrt(12)
\end{envconsolepythontex}
\end{minted}

\begin{envconsolepythontex}[largeur=14cm,centre=false]
	#console Python(tex) non centrée avec label
	from math import sqrt
	1+1
	sqrt(12)
\end{envconsolepythontex}

\begin{minted}[frame=lines,framesep=2mm,bgcolor=lightgray!5,fontsize=\footnotesize,tabsize=4]{tex}
\begin{envconsolepythontex}[largeur=14cm,label=false]
	#console Python(tex) centrée sans label
	table = [[1,2],[3,4]]
	table[0][0]
\end{envconsolepythontex}
\end{minted}

\begin{envconsolepythontex}[largeur=14cm,label=false]
	#console Python(tex) centrée sans label
	table = [[1,2],[3,4]]
	table[0][0]
\end{envconsolepythontex}

\end{document}