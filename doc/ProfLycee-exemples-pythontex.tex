% !TeX TXS-program:compile = txs:///arara
% arara: lualatex: {shell: no, synctex: yes, interaction: batchmode}
% arara: pythontex: {rerun: always} if found('pytxcode', 'PYTHONTEX#py')
% arara: lualatex: {shell: no, synctex: yes, interaction: batchmode} if found('pytxcode', 'PYTHONTEX#py')
% arara: lualatex: {shell: no, synctex: yes, interaction: batchmode} if found('log', '(undefined references|Please rerun|Rerun to get)')

\documentclass[french,a4paper,10pt]{article}
\def\PLver{2.7.3}
\usepackage[margin=1.5cm]{geometry}
\usepackage{ProfLycee}
\useproflyclib{pythontex}
\usepackage{babel}
\sisetup{locale=FR,output-decimal-marker={,},group-minimum-digits=4}
\usepackage{codehigh}

\begin{document}

\part*{ProfLycee (\PLver), Pythontex}

\section{Code \og Pythontex \fg{}}

\subsection{Préambule basique}

{\small \begin{codehigh}
\documentclass[french,a4paper,10pt]{article}
\usepackage{ProfLycee}
\useproflyclib{pythontex}                     % compilation spécifique
\end{codehigh}}

\subsection{Exemples}

{\small \begin{codehigh}
%Sortie par défaut
\begin{CodePythontex}{}
def valeur_absolue(x):
    "Renvoie la valeur absolue de x"
    #le petit test qui va bien
    if x > 0:
        return x
    else:
    return -x
\end{CodePythontex}
\end{codehigh}}

\begin{CodePythontex}{}
def valeur_absolue(x):
	"Renvoie la valeur absolue de x"
	#le petit test qui va bien
	if x > 0:
		return x
	else:
		return -x

#bla
#bla
#bla
\end{CodePythontex}

{\small \begin{codehigh}
%Sortie avec Style=Classique, Largeur=10cm, centré
\begin{CodePythontexAlt}[Largeur=10cm,Centre]{}
def valeur_absolue(x):
    "Renvoie la valeur absolue de x"
    #le petit test qui va bien
    if x > 0:
        return x
    else:
    return -x
\end{CodePythontexAlt}
\end{codehigh}}

\begin{CodePythontexAlt}[Largeur=10cm,Centre]{}
def valeur_absolue(x):
	"Renvoie la valeur absolue de x"
	#le petit test qui va bien
	if x > 0:
		return x
	else:
		return -x

#bla
#bla
#bla
\end{CodePythontexAlt}

\pagebreak

{\small \begin{codehigh}
%Sortie avec Style=Classique, Largeur=10cm
\begin{CodePythontexAlt}[Largeur=10cm]{}
def valeur_absolue(x):
    "Renvoie la valeur absolue de x"
    #le petit test qui va bien
    if x > 0:
        return x
    else:
    return -x
\end{CodePythontexAlt}
\end{codehigh}}

\begin{CodePythontexAlt}[Largeur=10cm]{}
def valeur_absolue(x):
	"Renvoie la valeur absolue de x"
	#le petit test qui va bien
	if x > 0:
		return x
	else:
		return -x
\end{CodePythontexAlt}

{\small \begin{codehigh}
%Sortie avec Style=Classique, Largeur=0.5\linewidth, centré, sans numéro
\begin{CodePythontexAlt}[Largeur=0.5\linewidth,Centre,Lignes=false]{}
def valeur_absolue(x):
    "Renvoie la valeur absolue de x"
    #le petit test qui va bien
    if x > 0:
        return x
    else:
    return -x
\end{CodePythontexAlt}
\end{codehigh}}

\begin{CodePythontexAlt}[Largeur=0.5\linewidth,Centre,Lignes=false]{}
def valeur_absolue(x):
	"Renvoie la valeur absolue de x"
	#le petit test qui va bien
	if x > 0:
		return x
	else:
		return -x
\end{CodePythontexAlt}

{\small \begin{codehigh}
%Sortie Moderne, Largeur=11cm, sans ligne
\begin{CodePythontex}[Largeur=11cm,Lignes=false]{}
def valeur_absolue(x):
    "Renvoie la valeur absolue de x"
    #le petit test qui va bien
    if x > 0:
        return x
    else:
    return -x
\end{CodePythontex}
\end{codehigh}}

\begin{CodePythontex}[Largeur=11cm,Lignes=false]{}
def valeur_absolue(x):
	"Renvoie la valeur absolue de x"
	#le petit test qui va bien
	if x > 0:
		return x
	else:
		return -x
\end{CodePythontex}

\pagebreak

\section{Console \og Pythontex \fg{}}

{\small \begin{codehigh}
%Déclaration d'une fonction python + librairie random pour utilisation ultérieure
\begin{pyconcode}
from random import randint

def valeur_absolue(x):
    "Renvoie la valeur absolue de x"
    #le petit test qui va bien
    if x > 0:
        return x
    else:
        return -x

\end{pyconcode}
\end{codehigh}}

\begin{pyconcode}
from random import randint

def valeur_absolue(x):
	"Renvoie la valeur absolue de x"
	#le petit test qui va bien
	if x > 0:
		return x
	else:
		return -x

\end{pyconcode}

{\small \begin{codehigh}
\begin{ConsolePythontex}{}
1+1
2**10
valeur_absolue(-3)
valeur_absolue(0)
valeur_absolue(5)
print(f"La valeur absolue de 5 est {valeur_absolue(5)}")
print(f"La valeur absolue de -4 est {valeur_absolue(-4)}")
\end{ConsolePythontex}
\end{codehigh}}

\begin{ConsolePythontex}{}
1+1
2**10
valeur_absolue(-3)
valeur_absolue(0)
valeur_absolue(5)
print(f"La valeur absolue de 5 est {valeur_absolue(5)}")
print(f"La valeur absolue de -4 est {valeur_absolue(-4)}")
\end{ConsolePythontex}

\pagebreak

{\small \begin{codehigh}
\begin{ConsolePythontex}[Largeur=12cm,Centre]{}
1+1
2**10
valeur_absolue(-3)
valeur_absolue(0)
valeur_absolue(5)
print(f"La valeur absolue de 5 est {valeur_absolue(5)}")
print(f"La valeur absolue de -4 est {valeur_absolue(-4)}")
liste = [randint(1,20) for i in range(10)]
print(liste)
print(max(liste), min(liste), sum(liste))
\end{ConsolePythontex}
\end{codehigh}}

\begin{ConsolePythontex}[Largeur=12cm,Centre]{}
1+1
2**10
valeur_absolue(-3)
valeur_absolue(0)
valeur_absolue(5)
print(f"La valeur absolue de 5 est {valeur_absolue(5)}")
print(f"La valeur absolue de -4 est {valeur_absolue(-4)}")
liste = [randint(1,20) for i in range(10)]
print(liste)
print(max(liste), min(liste), sum(liste))
\end{ConsolePythontex}

\end{document}