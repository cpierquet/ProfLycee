% !TeX document-id = {fc3853a9-0c9c-4d02-b763-6dc6c85d34db}
% !TeX TXS-program:compile = txs:///lualatex/[--shell-escape]

\documentclass[french,a4paper,10pt]{article}
\def\PLver{2.6.9}
\usepackage[margin=1.5cm]{geometry}
\usepackage{ProfLycee}
\useproflyclib{piton}
\usepackage[executable=python.exe]{pyluatex}
\usepackage{babel}
\sisetup{locale=FR,output-decimal-marker={,},group-minimum-digits=4}
\usepackage{codehigh}
%\usepackage[most]{tcolorbox}
%\newtcblisting{ShowCodeTeX}[1][]{colback=white,colframe=red!75!black,listing options={style=tcblatex},#1}

\begin{document}

\part*{ProfLycee (\PLver), Piton et Pyluatex}

\section{Code \og Piton \fg{}, indépendant de Pyluatex}

\subsection{Préambule basique}

{\small \begin{codehigh}
\documentclass[french,a4paper,10pt]{article}
\usepackage{ProfLycee}
\useproflyclib{piton}                           % lua
\end{codehigh}}

\subsection{Exemples}

{\small \begin{codehigh}
%Sortie par défaut
\begin{CodePiton}{}
def valeur_absolue(x):
    "Renvoie la valeur absolue de x"
    #le petit test qui va bien
    if x > 0:
        return x
    else:
    return -x
\end{CodePiton}
\end{codehigh}}

\begin{CodePiton}{}
def valeur_absolue(x):
	"Renvoie la valeur absolue de x"
	#le petit test qui va bien
	if x > 0:
		return x
	else:
		return -x
\end{CodePiton}

{\small \begin{codehigh}
%Sortie avec Style=Classique, Largeur=10cm
\begin{CodePiton}[Style=Classique,Largeur=10cm]{}
def valeur_absolue(x):
    "Renvoie la valeur absolue de x"
    #le petit test qui va bien
    if x > 0:
        return x
    else:
    return -x
\end{CodePiton}
\end{codehigh}}

\begin{CodePiton}[Style=Classique,Largeur=10cm]{}
def valeur_absolue(x):
	"Renvoie la valeur absolue de x"
	#le petit test qui va bien
	if x > 0:
		return x
	else:
		return -x
\end{CodePiton}

\pagebreak

{\small \begin{codehigh}
%Sortie avec Style=Classique, Sans Titre, Largeur=10cm, centré
\begin{CodePiton}[Style=Classique,Largeur=10cm,BarreTitre=false,Alignement=center]{}
def valeur_absolue(x):
    "Renvoie la valeur absolue de x"
    #le petit test qui va bien
    if x > 0:
        return x
    else:
    return -x
\end{CodePiton}
\end{codehigh}}

\begin{CodePiton}[Style=Classique,Largeur=10cm,BarreTitre=false,Alignement=center]{}
def valeur_absolue(x):
	"Renvoie la valeur absolue de x"
	#le petit test qui va bien
	if x > 0:
		return x
	else:
		return -x
\end{CodePiton}

{\small \begin{codehigh}
%Sortie avec Style=Classique, Largeur=0.5\linewidth, aligné à droite, sans Cadre, avec Filigrane
\begin{CodePiton}%
    [Style=Classique,Largeur=0.5\linewidth,Cadre=false,Alignement=flush right,Filigrane,Titre={Script}]{}
#environnement piton avec numéros de ligne, pleine largeur, style moderne
def valeur_absolue(x):
    "Renvoie la valeur absolue de x"
    #le petit test qui va bien
    if x > 0:
        return x
    else:
    return -x
\end{CodePiton}
\end{codehigh}}

\begin{CodePiton}[Style=Classique,Largeur=0.5\linewidth,Cadre=false,Alignement=flush right,Filigrane,Titre={Script}]{}
def valeur_absolue(x):
	"Renvoie la valeur absolue de x"
	#le petit test qui va bien
	if x > 0:
		return x
	else:
		return -x
\end{CodePiton}

{\small \begin{codehigh}
%Sortie Moderne, Largeur=11cm, avec Filigrane, aligné à gauche, sans ligne
\begin{CodePiton}[Largeur=11cm,Filigrane,Alignement=flush left,Lignes=false]{}
def valeur_absolue(x):
    "Renvoie la valeur absolue de x"
    #le petit test qui va bien
    if x > 0:
        return x
    else:
    return -x
\end{CodePiton}
\end{codehigh}}

\begin{CodePiton}[Largeur=11cm,Filigrane,Alignement=flush left,Lignes=false]{}
def valeur_absolue(x):
	"Renvoie la valeur absolue de x"
	#le petit test qui va bien
	if x > 0:
		return x
	else:
		return -x
\end{CodePiton}

\pagebreak

\section{Console \og Piton \fg{}, dépendant de Pyluatex}

\subsection{Préambule, avec le package pyluatex}

{\small \begin{codehigh}
\documentclass[french,a4paper,10pt]{article}
\usepackage{ProfLycee}
\useproflyclib{piton}
\usepackage[executable=python.exe]{pyluatex}    % lua + shell-escape
\end{codehigh}}

\subsection{Commande}

{\small \begin{codehigh}
\begin{ConsolePiton}[Options piton]<Clés>{Options tcbox}
...
...
\end{ConsolePiton}
\end{codehigh}}

\medskip

Les clés, à placer entre \texttt{<...>}, sont :

\begin{itemize}
	\item \textbf{\textsf{$\langle$Logo$\rangle$}} pour afficher un petit logo dans les \textit{titres} de la console REPL ; \hfill{}défaut : \textbf{\textsf{$\langle$true$\rangle$}}
	\item \textbf{\textsf{$\langle$Largeur$\rangle$}} pour spécifier la largeur de la console REPL ; \hfill{}défaut : \textbf{\textsf{$\langle$\textbackslash{}linewidth$\rangle$}}
	\item \textbf{\textsf{$\langle$Alignement$\rangle$}} pour spécifier l'alignement de la console REPL.\hfill{}défaut : \textbf{\textsf{$\langle$flush left$\rangle$}}
\end{itemize}

\subsection{Exemples}

{\small \begin{codehigh}
%Déclaration d'une fonction python + librairie random pour utilisation ultérieure
\begin{python}
from random import randint

def valeur_absolue(x):
    "Renvoie la valeur absolue de x"
    #le petit test qui va bien
    if x > 0:
        return x
    else:
        return -x
\end{python}
\end{codehigh}}

\begin{python}
from random import randint

def valeur_absolue(x):
	"Renvoie la valeur absolue de x"
	#le petit test qui va bien
	if x > 0:
		return x
	else:
		return -x
\end{python}

{\small \begin{codehigh}
\begin{ConsolePiton}{}
1+1
2**10
valeur_absolue(-3)
valeur_absolue(0)
valeur_absolue(5)
print(f"La valeur absolue de 5 est {valeur_absolue(5)}")
print(f"La valeur absolue de -4 est {valeur_absolue(-4)}")
\end{ConsolePiton}
\end{codehigh}}

\begin{ConsolePiton}{}
1+1
2**10
valeur_absolue(-3)
valeur_absolue(0)
valeur_absolue(5)
print(f"La valeur absolue de 5 est {valeur_absolue(5)}")
print(f"La valeur absolue de -4 est {valeur_absolue(-4)}")
\end{ConsolePiton}

\pagebreak

{\small \begin{codehigh}
\begin{ConsolePiton}<Largeur=11cm,Alignement=center,Logo=false>{}
1+1
2**10
valeur_absolue(-3)
valeur_absolue(0)
valeur_absolue(5)
print(f"La valeur absolue de 5 est {valeur_absolue(5)}")
print(f"La valeur absolue de -4 est {valeur_absolue(-4)}")
liste = [randint(1,20) for i in range(10)]
print(liste)
print(max(liste), min(liste), sum(liste))
\end{ConsolePiton}
\end{codehigh}}

\begin{ConsolePiton}<Largeur=11cm,Alignement=center,Logo=false>{}
1+1
2**10
valeur_absolue(-3)
valeur_absolue(0)
valeur_absolue(5)
print(f"La valeur absolue de 5 est {valeur_absolue(5)}")
print(f"La valeur absolue de -4 est {valeur_absolue(-4)}")
liste = [randint(1,20) for i in range(10)]
print(liste)
print(max(liste), min(liste), sum(liste))
\end{ConsolePiton}

\end{document}